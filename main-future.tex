\chapter{Conclusions and Future Work}
\label{chp-con}

In this dissertation, we aim to strengthen dependability of cloud-scale
distributed systems by addressing two new types of bugs in cloud systems,
distributed concurrency bugs (DC bugs) and scalability bugs. We have performed
bug studied to gain insights about the nature of these bugs, and we have also
advanced state of the art of system testing. This chapter concludes this
dissertation work and discuss future work in combating DC bugs and scalability
bugs.

\section{Conclusion}

\subsection{Distributed Concurrency Bugs}

The first problem we focus in this dissertation is DC bugs. We have conducted
in-depth study and created the largest and most comprehensive of DC bugs named
\taxdc. We categorize DC bugs in three dimensions. The first dimension is
triggering which is conditions that makes bugs happens. We studied timing
conditions and found four main timing patterns: order violation, atomicity
violation, fault timing, and reboot timing. We also studied input conditions
that are ingredients for bugs to surface. We found that most DC bugs will
surface only systems execute multiple protocols, and more than 50\% of bugs
surface in recovery protocols (\ie, the bugs surface only when there are
hardware failures).

The second dimension that we studied is errors and failures. We studied the
first errors that happen immediately after the bugs are triggered. We see half
of the bugs have local errors that is we can see the errors by observing only
triggering node, but half of them have global errors that require us to observe
the whole systems to notice the errors. Moreover, we looked into failure
symtomps induced by DC bugs and found that the bugs can lead to severe failures
like system downtime, operation failures, data loss/corruption/inconsistencies,
and performance degradation.

Lastly, the third dimension that we studied is fixes that are how developers fix
the DC bugs. We saw two main strategies to fix the bugs that are fixing the
timing and fixing the handling. For timing fixes, developers can do it globally or
locally (global synchronization or local synchronization). For handling fixes,
developers change the logic of message handling or fault handling such that the
systems still behave correctly.

Other than the bug study, we have introduced semantic-aware model checking
(SAMC) that is a white-box approach to model check the systems. SAMC prunes out
some executions because it knows that those executions are redundant with
previous executions it already tested by using semantic knowledge of target
systems. We have introduced four novel semantic-aware reduction policies, and
built \sampro and integrated it to three systems including Hadoop MapReduce,
Cassandra and ZooKeeper. On average, SAMC can find bugs 49x faster than other
states of the art.

\subsection{Scalability Bugs}

The second problem we focus in this dissertation is scalability bugs.
Scalability bugs are bugs that specific to cloud systems and not much attention
paid on them. We observed that scalability bugs are scale dependent and only
surface at extreme scale (\eg, hundreds of nodes). We found that although the
systems are designed to be scalable, but actual implementations can introduces
the bugs. We also saw that not all developers can afford large clusters to check
scalability of their code, and make bugs linger until the systems are deployed
on large scale.

Our observations highlight the need for scale checking that check the
implementation of the systems. Hence, we have introduced \sck, a scale-checking
methodology to allow developers colocate hundreds nodes on a single machine to
check their systems. We introduced four techniques to mitigate resource
contentions issue (\ie, CPU, memory, and threads) regarding to colocating
severals nodes in one machine. We adopted \sck to three systems including
Cassandra, Riak, and Voldemort, and were able to reproduce six old scalability
bugs with high accuracy.

\section{Future Work}

\subsection{Automated Semantic Extracting for SAMC}

\subsection{Full Automatic Scalability Checking Tool}

