
In the previous two chapters, we discuss about distributed concurrency bugs
that how we make sense of them and how we advance the state of the art of model
checking to unearth them faster. In this chapter, we will focus on
scalability aspect of cloud backend. As we mention in Section \ref{bg-sc}, the
trend of cloud distributed systems goes to horizontal scaling or scale-out systems.
On the positive side, scale surpasses the limit of a single machine in
meeting users' increasing demands of computing and storage, which led to
many inventions of ``cloud-scale'' distributed systems
\cite{Chang+06-BigTable, 
DeanGhemawat04-MapReduce, 
DeCandia+07-Dynamo,
Ghemawat+03-GoogleFS, 
Hindman+11-Mesos,
Verma+15-Borg}. The field has witnessed a
phenomenal deployment scale of such systems;
%
Netflix runs tens of 500-node Cassandra clusters \cite{RunningNetflix13},
% Running Netflix on Cassandra in the Cloud (youtube), Adriak Crockcroft
% https://www.youtube.com/watch?v=97VBdgIgcCU
Apple deploys a total of 100,000 Cassandra nodes \cite{WikiCassandra}, 
% https://en.wikipedia.org/wiki/Apache_Cassandra
and Yahoo! recently revealed the use of 40,000 Hadoop servers,
with a 4500-node cluster as the largest one \cite{LargestHadoop}.
% Http://www.techrepublic.com/article/why-the-worlds-largest-hadoop-installation-may-soon-become-the-norm/

% dark side, foe
On the negative side, scale creates new development and deployment issues.
Developers must ensure that their algorithms and protocol designs to be
scalable.  However, until real deployment takes place, scalability bugs in the
actual implementations are unforeseen.
% more and more
These scalability bugs are latent bugs that are scale-dependent; they only
surface in large-scale deployments, but not in small/medium-scale ones. Their
presence jeopardizes systems reliability and availability at scale.

In this chapter, we show our observations from our initial study on scalability
bugs that highlights an urgency in tackling scalability bugs and our pilot work
to introduce a single-machine testing methodology to check scalability of the
systems. Section \ref{sec-sck-observe} discusses about our initial study and shows our
observations toward scalability bugs; Section \ref{mot-state} discusses about the state
of the art; Section \ref{sec-sck-sck}-\ref{sec-sck-eval} explains our pilot work on scalability
bug checking named \sck and some evaluations.

