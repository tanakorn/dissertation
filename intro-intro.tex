As more data and computation move from local to cloud settings, cloud-scale
distributed systems such as scale-out storage systems \cite{Chang+06-BigTable,
DeCandia+07-Dynamo, Ghemawat+03-GoogleFS, Nightingale+12-FlatFDS}, computing
frameworks \cite{DeanGhemawat04-MapReduce, Murray+13-NaiadTimelyDataflow},
synchronization services \cite{Burrows06-Chubby, Hunt+10-ZooKeeperPaper}, and
cluster management services \cite{Hindman+11-Mesos, Kumar+13-Yarn} have become a
dominant backbone for many cloud services. Client-side software is getting
thinner and more heavily relies on the capability, reliability, and availability
of cloud systems. Users demand 24/7 dependability of cloud computing systems.
They must be accessible anytime and anywhere and not lose or corrupt users'
data, which means they must be reliable. Moreover, while user base continues
growing, they must be scalable also.

Unfulfilled dependability is costly. Some researchers estimate that 568 hours of
downtime at 13 well-known cloud services since 2007 to 2012 had an economic
impact of more than \$70 million~\cite{Essers12-70Million}. Others predict
worse: for every hour it is not up and running, a cloud service can take a hit
between \$1 to 5 million~\cite{Linthicum13-InfoworldCostOutages}.
Unfortunately, such cloud-scale distributed systems remain difficult to get
right. 
%
Cloud-scale distributed systems are getting more and more complex. New intricate
bugs continue to create dependability issues that cause major economic loss.
Guaranteeing dependability has proven to be challenging in these systems
\cite{Gunawi+11-FateDestini, Guo+11-Demeter, Wang+14-Exalt, Yang+09-Modist}.

In this proposal, we attempt to improve dependability of cloud-scale distributed
systems in two aspects, reliability and scalability. We are tackling this
challenge by answering these 2 questions, (1) What bugs that harm the
dependability, and (2) how do we test the systems to unearth these bugs so
developers can fix them? 
%
The first question is motivated by that we do not have comprehensive knowledge
about the bugs in distributed systems. There are many bug studies on
single-machine softwares \cite{Jin+12-PerformanceBugs,
Lu+08-ConcurrencyBugStudy, Palix+11-FaultsInLinux,
Sahoo+10-StudyBugsServerSoftware}, yet there are few formal bug studies on
distributed-systems softwares; they did not study in a great number and across
multiple types of systems \cite{Li+13-ScopeBugStudy, Xiao+14-NonDetMR}. We
believe that we need comprehensive understanding about cloud bugs to combat
them.

For the second question, we are motivated by the fact that in the past decade,
systems community has developed many testing techniques
\cite{Gunawi+11-FateDestini, Guo+11-Demeter, Wang+14-Exalt, Yang+09-Modist} to
find bugs in distributed systems, but these techniques still have limitations.
For example, \fate\ \cite{Gunawi+11-FateDestini} tests reliability of systems by
injecting faults, but it does not address concurrency in distributed systems.
\modist, which is a model checker, addresses concurrency, but it cannot work in
reasonable time when injecting multiple faults. Or Exalt, which is a framework
to test scalability, cannot be applied to CPU-intensive systems. 

We propose how to further the current testing techniques beyond the limitations
in this proposal. The proposal is arranged in this order: chapter \ref{chp-bg}
explains the problem being solved in detail and discusses related work, chapter
\ref{chp-plan} shows our research plans, and chapter \ref{chp-con} gives a
conclusion.
%
The proposal is a fusion of our previous work and our on-going work. It includes
cloud bug studies \cite{Gunawi+14-Cbs, Leesatapornwongsa+16-TaxDC},
semantic-aware model checking \cite{Leesatapornwongsa+14-Samc}, and scale check
methodology.

