``\textit{Cloud computing}'' has been given many definitions from many companies
and experts \cite{TwentyoneCloudDef, IBMCloudDef, PCMagCloudDef,
Foster+08-CloudAndGrid}. These definitions are different in details, but they
have some common characteristics; they are on-demand internet-based services
that can scale to fit increasing users, and users pay only for their use.
%
Cloud computing help users (from end users to organizational users) reduce the
capital investment in hardware that is mostly underutilized
\cite{Hayes+08-CloudComputing} and help business moves faster
\cite{Marston+11-CloudBusiness}. We see a trend that users are moving their data
and computation from local machines and in-house datacenters to the cloud
\cite{AdobeCloudStat, AWSCustomer, GmailStat, GoogleDriveStat, DropboxStat,
AstroInCloud, FacebookStat, Luo+16-BigDataBioResearch}.

This trend makes client-side software get thinner and more heavily rely on the
cloud services, thus the services are expected to be 24/7 dependable. Cloud
services must be accessible anytime and anywhere and not lose or corrupt users
data, and scale as user base continues to grow \cite{Buyya+09-Cloud5thUtil}.
%
Unfulfilled dependability is costly. Some researchers estimate that 568 hours of
downtime at 13 well-known cloud services since 2007 to 2012 had an economic
impact of more than \$70 million \cite{Essers12-70Million}. Others predict
worse: for every hour it is not up and running, a cloud service can take a hit
between \$1 to 5 million \cite{Linthicum13-InfoworldCostOutages}.

Unfortunately, proving cloud services' dependability is challenging. Behind the
cloud computing, it is backed by large sophisticated distributed software stack
\cite{Burrows06-Chubby, Chang+06-BigTable, Chapin+95-Hive, Corbett+12-Spanner,
DeanGhemawat04-MapReduce, DeCandia+07-Dynamo, Ghemawat+03-GoogleFS,
Hunt+10-ZooKeeperPaper, Lakshman+09-Cassandra, Melnik+10-DremelInteractive,
Zaharia+12-RDD} that is running on top of large-scale cluster \xxx{citation}.
Such cloud distributed systems remain difficult to get right because they need
to address data races among machines, complex failures that randomly happen,
tremendous user requests, and much more issues that caused from cloud computing
infrastructure.

Data races are known to be a core problem in any concurrent software systems.
Unlike non-distributed software, cloud distributed systems are subject to not
only local concurrency bugs, which basically come from thread interleaving, but
also distributed concurrency bugs, which come from inter-node message
interleaving.  Moreover, cloud hardware is built from commodity hardware that
failures can happen at anytime and can be very complex. The timing of these
hardware failures plus message interleaving makes it hard to handle the
concurrency correctly.

Moreover, the size of cloud users is tremendous and cloud service providers need
to guarantee service quality (\ie, availability and performance) to their users.
The providers need to ensure that their capabilities can satisfy the current
users and also make sure there is no glitch when users are growing. Cloud
providers normally employ large-scale systems to achieve high aggregate
capabilities, but large-scale systems are challenging to build and costly to
test their correctness.

Addressing these challenges makes the systems getting more complex. New
intricate bugs continue to happen and create dependability problems.
Guaranteeing dependability has proven to be challenging in these systems
\cite{Gunawi+11-FateDestini, Guo+11-Demeter, Wang+14-Exalt, Yang+09-Modist}.
This raises a vital question: ``{\em how can we make cloud-scale distributed
systems reaching ideal dependability?}'' We try to answer this question by
focusing on the problems of distributed concurrency bugs and scalability bugs.
These two are critical problems because they are novel issues that occur in
cloud environment only and not many works addressing them.  The following
sections discuss our contributions to address the challenges.

