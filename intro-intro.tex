``\textit{Cloud computing}'' has been given many definitions from many companies
and experts \cite{TwentyoneCloudDef, IBMCloudDef, PCMagCloudDef,
Foster+08-CloudAndGrid}. These definitions are different in details, but they
have some common characteristics; they are on-demand internet-based services
that can scale to fit increasing users' requirements, and users pay only for
their use.
%
Cloud computing help users (from end users to organizational users) reduce the
capital investment in hardware and can make busisness move faster (\xxx{need
citation}). We see a trend that users are moving their data and computation from
local machines and in-house datacenters to the cloud \cite{AdobeCloudStat,
AWSCustomer, GmailStat, GoogleDriveStat, DropboxStat, FacebookStat} \xxx{add
more citation from scientific world}.

This trend makes client-side software get thinner and more heavily rely on the
cloud services, thus the services are expected to be 24/7 dependable. Cloud
services must be accessible anytime and anywhere and not lose or corrupt users
data (reliability), and scale as user base continues to grow (scalability)
\cite{Buyya+09-Cloud5thUtil}.
%
Unfulfilled dependability is costly. Some researchers estimate that 568 hours of
downtime at 13 well-known cloud services since 2007 to 2012 had an economic
impact of more than \$70 million \cite{Essers12-70Million}. Others predict
worse: for every hour it is not up and running, a cloud service can take a hit
between \$1 to 5 million \cite{Linthicum13-InfoworldCostOutages}.

Unfortunately, proving cloud services' dependability is challenging. Behind the
cloud computing, it is backed by large sophisticated distributed software stack
\cite{Burrows06-Chubby, Chang+06-BigTable, Chapin+95-Hive, Corbett+12-Spanner,
DeanGhemawat04-MapReduce, DeCandia+07-Dynamo, Ghemawat+03-GoogleFS,
Hunt+10-ZooKeeperPaper, Lakshman+09-Cassandra, Melnik+10-DremelInteractive,
Zaharia+12-RDD} that is running on top of large-scale cluster \xxx{citation}.
Such cloud distributed systems remain difficult to get right because they need
to address data races among machines, complex failures that randomly happen,
tremendous user requests, and much more issues that caused from cloud computing
infrastructure.

Data races are a well-known problem in any concurrent software systems. Unlike
non-distributed software, cloud distributed systems are subject to not only
local concurrency bugs, which basically come from thread interleaving, but also
distributed concurrency bugs, which come from inter-node message interleaving. 
Moreover, cloud hardware is built from commodity hardware that failures can
happen at anytime and can be very complex. The timing of these hardware failures
plus message interleaving makes it hard to handle the concurrency correctly.

\xxx{talk about tremenduos user request}

Addressing these challenges makes the systems getting more complex. New
intricate bugs continue to happen and create dependability problems.
Guaranteeing dependability has proven to be challenging in these systems
\cite{Gunawi+11-FateDestini, Guo+11-Demeter, Wang+14-Exalt, Yang+09-Modist}.
This raises a vital question: ``{\em how can we make cloud-scale distributed
systems reaching ideal dependability?}'' We try to answer this question by
focusing on the problems of distributed concurrency bugs and scalability bugs.
These two are critical problems because they are novel issues that occur in
cloud environment only and not many works addressing them.  The following
sections discuss our contributions to address the challenges.

\if 0

As more data and computation move from local to cloud settings, cloud-scale
distributed systems such as scale-out storage systems \cite{Chang+06-BigTable,
DeCandia+07-Dynamo, Ghemawat+03-GoogleFS, Nightingale+12-FlatFDS}, computing
frameworks \cite{DeanGhemawat04-MapReduce, Murray+13-NaiadTimelyDataflow},
synchronization services \cite{Burrows06-Chubby, Hunt+10-ZooKeeperPaper}, and
cluster management services \cite{Hindman+11-Mesos, Kumar+13-Yarn} have become a
dominant backbone for many cloud services. Client-side software is getting
thinner and more heavily relies on the capability, reliability, and availability
of cloud systems. Users demand 24/7 dependability of cloud computing systems.
They must be accessible anytime and anywhere and not lose or corrupt users'
data, which means they must be reliable; they have to provision fast and stable
response time, which means they need stable performance; and while user base
continues growing, they must be scalable also.

Unfulfilled dependability is costly. Some researchers estimate that 568 hours of
downtime at 13 well-known cloud services since 2007 to 2012 had an economic
impact of more than \$70 million~\cite{Essers12-70Million}. Others predict
worse: for every hour it is not up and running, a cloud service can take a hit
between \$1 to 5 million~\cite{Linthicum13-InfoworldCostOutages}.
Unfortunately, such cloud-scale distributed systems remain difficult to get
right. 
%
Cloud-scale distributed systems are getting more and more complex. New intricate
bugs continue to create dependability issues that cause major economic loss.
Guaranteeing dependability has proven to be challenging in these systems
\cite{Gunawi+11-FateDestini, Guo+11-Demeter, Wang+14-Exalt, Yang+09-Modist}.

%\section{Dependability Research}

In this proposal, we attempt to improve dependability of cloud-scale distributed
systems. We are tackling this challenge by answering these 2 questions, (1) What
bugs that harm the dependability?, and (2) how do we test the systems to unearth
these bugs so developers can fix them? 
%
The first question is motivated by that we do not have comprehensive knowledge
about the bugs in distributed systems. There are many bug studies on
single-machine softwares \cite{Jin+12-PerformanceBugs,
Lu+08-ConcurrencyBugStudy, Palix+11-FaultsInLinux,
Sahoo+10-StudyBugsServerSoftware}, yet there are few formal bug studies on
distributed-systems softwares; they did not study in a great number and across
multiple types of systems \cite{Li+13-ScopeBugStudy, Xiao+14-NonDetMR}. We
believe that we need comprehensive understanding about cloud bugs to combat
them.

For the second question, we are motivated by the fact that in the past decade,
systems community has developed many testing techniques
\cite{Gunawi+11-FateDestini, Guo+11-Demeter, Wang+14-Exalt, Yang+09-Modist} to
find bugs in distributed systems, but these techniques still have limitations.
For example, \fate\ \cite{Gunawi+11-FateDestini} tests reliability of systems by
injecting faults, but it does not address concurrency in distributed systems.
\modist, which is a model checker, addresses concurrency, but it cannot work in
reasonable time when injecting multiple faults. Or Exalt, which is a framework
to test scalability, cannot be applied to CPU-intensive systems. 

We choose to start dependability research on two aspects, reliability and
scalability.
%
For reliablity, we find that one unsolved reliability problem in cloud systems
is ``{\em distributed concurrency (DC) bugs}''. DC bugs are caused  by
non-deterministic order of distributed events such as message arrivals, faults,
and reboots. Cloud systems execute multiple complicated distributed protocols
concurrently (\eg, serving users' requests, operating some background tasks, and
combined with untimely hardware failures). The possible interleavings of the
distributed events are not completely envisioned by developers and some
interleavings might not be handled properly. The buggy distributed interleavings
can cause catastrophic failures such as data loss, data inconsistencies and
downtimes. Our effort to tackle reliability issues will concentrate on DC
bugs.

And for scalability, we see that most of the work \cite{Calotoiu+13-ApmScaleBug,
Laguna+15-DebugAtScale, Shudler+15-ExascaleLib, Wang+14-Exalt, Zhou+11-Vrisha,
Zhou+13-Wukong} focuses on the data path, mainly to validate the scalability of
read/write operations (linear throughput or stable latency as the cluster
scales). But scalability correctness however is not merely about the data path.
Distributed systems are full of ``control paths'' such as bootstrapping,
rebalancing, and adding/decommissioning nodes (scaling out/down). These
management protocols must modify cluster-wide metadata that lives in each
node in the system (\eg, ring partition table) to decide how data flows in
the cluster. Unfortunately, control path correctness is often overlooked, so we
aim our attention to ``{\em control-plane scalability bugs}'' in this proposal.

We propose how to further the current testing techniques beyond the limitations
in this proposal. The proposal is arranged in this order: chapter \ref{chp-bg}
explains the problem being solved in detail and discusses related work, chapter
\ref{chp-detail} shows our research in detail, and chapter \ref{chp-con} gives a
conclusion.
%
The proposal is a fusion of our previous work and our on-going work. It includes
cloud bug studies \cite{Gunawi+14-Cbs, Leesatapornwongsa+16-TaxDC},
semantic-aware model checking \cite{Leesatapornwongsa+14-Samc}, and scale check
methodology.

\fi
