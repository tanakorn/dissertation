


\subsection{Evaluation Scope (vs. Other Work)}
\label{eval-other}

%We now compare the scale of our evaluation with Exalt's
%\cite{Wang+14-Exalt} and DieCast's \cite{Gupta+08-DieCast} in various
%dimensions:
%
\begin{enumerate}
\item {\bf Colocation factor:} DieCast is primarily evaluated with 10
nodes/machine.  Exalt runs up to 100 nodes per 16-core machine
% (limited by CPU bottlenecks).  
\sck can achieve up to \maxCF colocation factor.
However, in our view, Exalt and \sck \textit{complement each other} as
they target different types of bugs (data- vs. compute-intensive).
%
\item {\bf \#Target systems:} Exalt is integrated to two master-slave
systems (HBase and HDFS), DieCast to 3 systems (BitTorrent, RUBis, and
ISaaC), and \sck to three P2P key-value stores.
%
\item {\bf \#Target protocols:} Exalt and DieCast test write protocols and
\sck tests \numProt control-path protocols.
%
\item {\bf \#Bugs:} DieCast did not reproduce any bugs; it mainly
evaluates throughput/latency accuracy.
%
Exalt discussed 6 bugs in total (5 out of 6 are bugs in the Namenode, the
non-emulated node; \sec\ref{mot-state}).
%
\sck reproduced 6 bugs in emulated P2P nodes.
%
\item {\bf Types of bugs:} While most work focus on data-plane bugs
(\sec\ref{mot-control}, \sec\ref{mot-state}), \sck focuses on
control-plane protocols which are mainly about cluster stability (no
flapping, eventually balanced, \etc).

\end{enumerate}
