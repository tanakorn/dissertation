


\section{Speed in Finding Old Bugs}
\label{eval-oldbugs}







\begin{figure}

% MR-5505: CM
\begin{center}
\fbox{
\begin{minipage}{5.5in}
{\bf \mr{5505}:}
\begin{enumerate}
\item A job finishes,
\item Application manager (AM) sends a ``remove-app'' message to Resource Manager (RM),
\item RM receives the message,
\item AM is unregistering,
\item \fev{RM crashes} before completely processes the message,
\item AM finishes unregistering,
\item \fev{RM reboots} and reads the old state file,
\item RM thinks the job has never started and runs the job again.
\end{enumerate}
%\ev{(1)} A job finishes,
%\ev{(2)} Application manager (AM) sends a ``remove-app'' message
%to Resource Manager (RM),
%\ev{(3)} RM receives the message,
%\ev{(4)} AM is unregistering,
%\ev{(5)} \fev{RM crashes} before completely processes the message,
%\ev{(6)} AM finishes unregistering,
%\ev{(7)} \fev{RM reboots} and reads the old state file,
%\ev{(8)} RM thinks the job has never started and runs the job again.
\end{minipage}
}
\end{center}


% CASS-3395: Hinted handoff
\vten
\begin{center}
\fbox{
\begin{minipage}{5.5in}
{\bf \cs{3395}}
\begin{enumerate}
\item Three nodes N1-3 started and formed a ring,
\item Client writes data,
\item \fev{N3 crashes},
\item Client updates the data via N1; N3 misses the update,
\item \fev{N3 reboots},
\item N1 begins the hinted handoff process,
\ev{(7)} Another client reads the data with strong consistency via N1 as
a coordinator,
\item N1 and N2 provide the updated value, but N3 still provides
the stale value,
\item The coordinator gets ``confused'' and returns the stale value to
the client!
\end{enumerate}
%\ev{(1)} Three nodes N1-3 started and formed a ring,
%\ev{(2)} Client writes data,
%\ev{(3)} \fev{N3 crashes},
%\ev{(4)} Client updates the data via N1; N3 misses the update,
%\ev{(5)} \fev{N3 reboots},
%\ev{(6)} N1 begins the hinted handoff process,
%\ev{(7)} Another client reads the data with strong consistency via N1 as
%a coordinator,
%\ev{(8)} N1 and N2 provide the updated value, but N3 still provides
%the stale value,
%\ev{(9)} The coordinator gets ``confused'' and returns the stale value to
%the client!
\end{minipage}
}

\end{center}


% MR 5863 (new)




\mycaption[Complexity of Deep Bugs]{code-oldbugs}{Complexity of Deep Bugs}{
%
Above are two sample deep bugs in Hadoop and Cassandra.  A sample
for ZooKeeper was shown in the introduction
(\sec\ref{sec-intro}).  Deep bugs are complex to reproduce; crash and
reboot events must happen in a specific order within a long sequence of
events (there are more events behind the events we show in the
bug descriptions above).
To see the high degree of complexity of other old bugs
that we reproduced, interested readers can click the
  issue numbers (hyperlinks) in Table~\ref{tab-oldbugs}.
%
}
%\vminfive
\end{figure}



\if 0

% CASS-3626: Gossiper
\vten
\fbox{
\begin{minipage}{2.95in}
{\bf \cassandra{3626}}
\ev{(1)} Three nodes N1, N2, and N3 start, sending join messages,
\ev{(2)} N1 receives message from N3,
\ev{(3)} N2 receives message from N3,
\ev{(4)} \fev{N3 crashes before} it detects its peers,
\ev{(5)} N1 receives message from N2,
\ev{(6)} N2 receives message from N1,
\ev{(7)} \fev{N2 crashes},
\ev{(8)} N1 detects N2 and N3 are down,
\ev{(9)} \fev{N3 reboots},
\ev{(10)} N3 wrongly believes N2 is up.
\end{minipage}
}
\fi



This section evaluates the speed of SAMC vs. state-of-the-art
techniques in finding old deep bugs.  In total, we have reproduced
\numOldDeepBugs\ old deep bugs (\numZkDeepBugs\ in ZooKeeper,
\numMrDeepBugs\ in Hadoop, and \numCsDeepBugs\ in Cassandra).
Figure~\ref{code-oldbugs} illustrates the complexity of the deep bugs
that we reproduced.



\def \fth {\uuu 5000}



\begin{table*}[!hbt]
\begin{center}
{\small
%---------------------------------
\begin{tabular}{l|l|ccc|rrrr|rrr} 

 {} & {} &
 {} & {} & {} &
\multicolumn{4}{c|}{{\bf \#Executions }} &
\multicolumn{3}{c}{{\bf Speed-up of SAMC vs. }} \\


{\bf Old Issue\#}  &  {\bf Protocol} & 
{\bf E} & {\bf C} & {\bf R} &
\bf{bDP} & \bf{RND} & \bf{rDP} & \bf{SAMC} &
\bf{bDP} & \bf{RND} & \bf{rDP} \\

\hline
\zk{335}   &   ZAB  & 120 & 3 & 3  &  \fth &  1057 &  \fth &   117 &  \uu 43 &      9 & \uu 43 \\
\zk{790}   &   ZLE  &  21 & 1 & 1  &    14 &   225 &    82 &     7 &       2 &     32 &     12 \\
\zk{975}   &   ZLE  &  21 & 1 & 1  &   967 &    71 &   163 &    53 &      18 &      1 &      3 \\
\zk{1075}  &   ZLE  &  25 & 3 & 2  &  1081 &    86 &   250 &    16 &      68 &      5 &     16 \\
\zk{1419}  &   ZLE  &  25 & 3 & 2  &   924 &  2514 &   987 &   100 &       9 &     25 &     10 \\
\zk{1492}  &   ZLE  &  31 & 1 & 0  &  \fth &  \fth &  \fth &   576 &   \uu 9 &  \uu 9 &  \uu 9 \\
\zk{1653}  &   ZAB  &  60 & 1 & 1  &   945 &  3756 &  3462 &    11 &     86  &    341 &    315 \\
\hline
\mr{4748}  &    SE  &  25 & 1 & 0  &    22 &     6 &     6 &     4 &       6 &      2 &      2 \\
\mr{5489}  &    CM  &  20 & 2 & 1  &  \fth &  \fth &  \fth &    53 &  \uu 94 & \uu 94 & \uu 94 \\
\mr{5505}  &    CM  &  40 & 1 & 1  &  1212 &  \fth &  1210 &    40 &      30 & \uu125 &     30 \\
\hline
\cs{3395}  & RW+HH  &  25 & 1 & 1  &  2552 &   191 &   550 &   104 &      25 &      2 &      5 \\
\cs{3626}  &    GS  &  15 & 2 & 1  &  \fth &  \fth &  \fth &    96 &  \uu 52 & \uu 52 & \uu 52 \\
\end{tabular}
}
%\vminten
%---------------------------------
\end{center}
\mycaption{tab-oldbugs}{SAMC Speed in Finding Old Bugs}{
%
``E'', ``C''
and ``R'' represent the number of events, crashes, and reboots
necessary to hit the bug.  The numbers in the middle four columns
represent the number of executions to hit the bug across different
policies.  ``bDP'', ``RND'', and ``rDP'' stand for black-box DPOR (in
\modist), random, and random + black-box DPOR respectively.
%Numbers marked with ``\nu'' imply that the experiments are still running
%at the time of this paper submission and the bug has not been reached.
We stop at 5000 executions (around 2 days) if the bug
cannot be found (labeled with ``\uuu'').  
Thus, speed-up numbers marked with
``\uu'' are potentially much higher.  
%
}

%\vminfive

\end{table*}

\if 0
\zk{1653}  &    --  & --- & 1 & 1  &    -- &    -- &    -- &    -- &     --- & --- & --- \\
\fi


Table~\ref{tab-oldbugs} shows the result of our comparison.  We
compare SAMC with basic techniques (DFS and Random) and advanced
state-of-the-art techniques such as black-box DPOR (``bDP'') and
Random+bDP (``rDP'').  Black-box DPOR is the \modist-style of DPOR
(\sec\ref{mot-state}).  We include Random+DPOR to mimic the way
\modist\ authors found bugs faster (\sec\ref{mot-summ}).  The table
shows the number of executions to hit the bug.  As a note,
software model checking with the inclusion of failures
takes time (back-and-forth communications between the target system
and the dmck server, killing and restarting system processes multiple
times, restarting the whole system from a clean state, \etc).  On
average, each execution runs for \numAvgExecTime\ seconds and involves
a long sequence of 20-120 events including the necessary crashes and
reboots to hit the bug.  We do not show the result of running DFS
because it never hits most of the bugs.  


Based on the result in Table~\ref{tab-oldbugs}, we make several
conclusions.
%
First, with SAMC, we prove that smart systematic approaches can reach
to deep bugs quickly.  We do not need to revert to randomness or
incorporate checkpoints.  As a note, we are able to reproduce every
deep bug that we picked; we did not skip any of them.
%
(Hunting more deep bugs is possible, if needed).

Second, SAMC is one to two orders of magnitude faster compared to
state-of-the-art techniques.  Our speed-up is up to
\numMaxBugSpeedUp\ (\numAvgBugSpeedUp\ on average).  But most
importantly, there are bugs that other techniques cannot find even
after 5000 executions (around 2 days). Here, SAMC's speed-up factor is
potentially much higher (labeled with ``\uu'').  Again, in the context of
dmck (a process of hours/days), large speed-ups matter.  In many
cases, state-of-the-art policies such as bDP and rDP cannot reach the
bugs even after very long executions.  The reasons are the two
problems we mentioned earlier (\sec\ref{mot-summ}).  Our
micro-analysis (not shown) confirmed our hypothesis that non-SAMC
policies frequently make redundant crash/reboot injections and event
re-orderings that anyway lead to insignificant state changes.

Third, Random is truly ``random''.  Although many previous dmcks
embrace randomness in finding bugs~\cite{Killian+07-LifeDeathMaceMC,
  Yang+09-Modist}, when it comes to failure-induced bugs, we have a
different experience.  Sometimes Random is as competitive as SAMC
(\eg, \zkb{975}), but sometimes Random is much slower (\eg,
\zkb{1419}), or worse Random sometimes did not hit the bug (\eg,
\zkb{1492}, \mrb{5505}).  We find that some bugs require crashes
and/or reboots to happen at very specific points, which is
probabilistically hard to reach with randomness.  With SAMC, we show
that being systematic and semantic aware is consistently effective.

