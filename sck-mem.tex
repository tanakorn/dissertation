

\subsection{Pre-Memoization and Replay Time}
\label{eval-mem}



The ``$T_m$'' and ``$T_{pil}$'' columns in Table \ref{tab-bugs} quantifies
the duration of the pre-memoization (\stestm) and PIL-based replay (\stestp)
stages when $N$$\geq$$256$.
%
For example, for CPU-intensive bugs such as \caone, the pre-memoization
time takes 2 hours while the PIL-based replay is only 15 minutes (similar
to the real deployment test); for \riakone, it is 6 vs. 2 hours.
%
Pre-memoization does not necessarily take $N$$\times$ longer time because
one node only consumes 2 cores (while the machine has 16 cores) and also
not every node is busy all the time.
%
For bugs that are caused by both processing and IO time, $T_m$ is similar
to $T_{pil}$ (\eg, in \catwo).
%
To summarize, with fast PIL-infused replays, developers can have 
sufficient time budget to debug the buggy protocol numerous times as needed to
discover the root cause.
%
Regarding the pre-memoization database file, the largest one created was
3 GB with 50000 input-output pairs.


% \input{fig-eval-mem}


