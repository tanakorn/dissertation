

\section{Deep Bugs}
\label{mot-deep}


\if 0
\tl{response to reviewer E, regarding fix time for each bug, and
encounter dep bugs that developers don't think are worth fixing.
    COMMENT: We can show the time since an issue was reported until it was
    closed, but we do not think that it tells us important information.
    Longer fix time does not mean it is harder to fix, it might mean
    developers asked for more information (e.g. Log files) and reporters
    need some time to gather that or wait until bugs happen again.  For the
    bugs that developers do not think they are worth to fix, yes we have
    seen some. The example of these bugs could be the wrong state of
    systems that eventually be detected and fixed by some error handle
    code. This kind of bugs makes a few nodes down for few minutes (before
    alive again) that developers think it is not that bad.}
\fi


% 94
% 54 total

% bug study
To understand the unique reliability challenges faced by cloud
systems, we performed a study of reliability bugs of three popular
cloud systems: ZooKeeper~\cite{Hunt+10-ZooKeeperPaper}, Hadoop
MapReduce~\cite{Kumar+13-Yarn}, and
Cassandra~\cite{Lakshman+09-Cassandra}.  We scanned through thousands
of issues from their bug repositories.  We then tagged complex
reliability bugs that can only be caught by a dmck (\ie, bugs that can
occur only on specific orderings of events).  We found 94
dmck-catchable bugs.\footnote[1]{Since this is a manual effort, we
  might miss some bugs.  We also do not report ``simple'' bugs (\eg,
  error-code handling) that can be caught by unit tests.}  Our major
finding is that 50\% of them are deep bugs (require complex
re-ordering of not only messages but also crashes and reboots).



% results, and observations
Figure~\ref{fig-deepbugs} lists the deep bugs found from our bug
study.  Many of them were induced by multiple crashes and reboots.
Worse, to reproduce the bugs, crash and reboot events must happen in a
specific order within a long sequence of events (\eg, the example bug
in \sec\ref{sec-intro}).  Deep bugs lead to harmful consequences (\eg,
failed jobs, node unavailability, data loss, inconsistency,
corruption), but they are hard to find.  We observe that since there
is no dmck that helps in this regard, deep bugs are typically found in
deployment (via logs) or manually, then they get fixed in few
weeks, but afterwards as code changes continuously, new deep bugs
tend to surface again.
% , as confirmed by cloud developers~\cite{ClouderaPC}.

