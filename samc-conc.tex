\section{Conclusion}

In this chapter, we present semantic-aware model checking, a white-box principle
that takes simple semantic information of the target system and incorporates
that knowledge in state-space reduction policies by skipping redundant
executions. We find that simple semantic knowledge can assist dmck greatly; the
semantic are from \textit{event processing semantic} (\ie, how messages,
crashes, and reboots are processed by the target system). To help testers
extract such information from the target system, we provide \textit{generic
event processing patterns}, patterns of how messages, crashes, and reboots are
processed by distributed systems in general.

We introduce four novel semantic-aware reduction policies: 
\begin{enumerate}

\item {\bf Local-Message Independence} (LMI) reduces re-orderings of intra-node
messages. 

\item {\bf Crash-Message Independence} (CMI) reduces re-orderings of crashes
among outstanding messages.  

\item {\bf Crash Recovery Symmetry} (CRS) skips crashes that lead to same
recovery behaviors.

\item {\bf Reboot Synchronization Symmetry} (RSS) skips reboots that lead to
same synchronization actions.  

\end{enumerate}
Our reduction policies are {\em generic}; they are applicable to many
distributed systems. SAMC users (\ie, testers) only need to feed the policies
with short {\em protocol-specific rules} that describe event independence and
symmetry specific to their target systems. And SAMC is purely systematic; it
does not incorporate randomness or bug-specific knowledge.

