\section{Fixes}
\label{sec-fix}

We next analyze bug patches to understand developers' fix strategies.
In general, we find that DC bugs can be fixed by either disabling the
triggering timing or changing the system's handling to that timing
({\em fix timing} vs. {\em fix handling}).  The former
prevents concurrency with extra synchronization and the latter allows
concurrency by handling untimely events properly.  Since message
timing bugs are fixed quite differently from fault timing bugs, we
separate them below.

% -------------------------------------------------
\subsection{Message Timing Bug Fixes}
\label{fix-msg}




The left half of Table \ref{tab:fix-msg} shows that
only one fifth of message timing bugs are fixed by disabling the
triggering timing, through either
 global or local synchronization.
%
Only a couple of bugs are fixed through extra {\em global
  synchronization}, mainly due to its complexity and communication
cost.  For example, to prevent a triggering pattern \lbp\ss\mab\ in
Figure \ref{pat}f, \mr{5465}'s fix {\em adds} a monitor on A\sub{RM}
to wait for \mba\sub{done} message from B\sub{AM} after B finishes
with its local computation (\lbp); the result is
\lbp-\mba-\mab\ global serialization.
%
More often, the buggy timing is disabled through {\em local
  synchronization}, such as re-ordering message sending operations
within a single node.  For example, \hb{5780}'s fix for
\mab\ss\mbc\ race in Figure \ref{pat}c forces the sending of
\mbc\ request at B to wait for the receipt of \mab; the result is
\mab-\mbc\ local serialization at B.




\begin{table}[t]
\small
\centering
\begin{tabular}{lccccccc}
\toprule
   & \multicolumn{2}{c}{Fix Timing} && \multicolumn{4}{c}{Fix Handling}\\
\cmidrule{2-3}
\cmidrule{5-8}
   & Glob & Loc & & Ret & Ign & Acc & Misc \\
\midrule
CA & 0   & 0 &&0 &1 & 3 & 4 \\
HB & 2   & 7 &&2 &1 & 7 & 3 \\
MR & 2   & 8 &&2 &7 & 8 & 3 \\
ZK & 0   & 4 &&0 &3 & 0 & 1 \\
\midrule
All& 4   & 19&&4 &12& 18& 11\\
\bottomrule
\end{tabular}
\mycaption{tab:fix-msg}{Fix strategies for message timing bugs 
(\sec{\ref{fix-msg}})}
{Some bugs require more than one fix strategy.}
\end{table}



\if 0
\jirafootnote{fix-msg}{tab:fix-msg}{
\spa \hb{4729}, \mr{5465};                       % FixMsgTimeGlobal
\spb \hb{5780}, \mr{4099}, \zk{1144};            % FixMsgTimeLocal
\spc \hb{8940}, \mr{3274};                       % FixMsgHandRetry
\spd \ca{2371}, \hb{6227}, \mr{4252}, \zk{1208}; % FixMsgHandIgnore
\spe \ca{6013}, \hb{6070}, \mr{4637};     % FixMsgHandAcc
\spf \ca{5631}, \hb{6537}, \mr{3858}, \zk{962};     % FixMsgHandOth
}
\fi

The right half of Table \ref{tab:fix-msg} shows that fix handling is
more popular.  Fix handling fortunately can be simple;  many fixes
do {\em not} introduce brand-new computation logic into
the system, which can be done in three ways.
% 
First, the fix can handle the untimely message by simply {\em
  retrying} it at a later time (as opposed to ignoring or
accepting it incorrectly).  For example, to handle \mbc\ss\mac\ race
in Figure \ref{pat}a, \mr{3274} retries the unexpectedly-early
\mac\sub{kill} message at a later time, right after the
to-be-killed task starts.
%
Second, the fix can simply {\em ignore} the message (as opposed
to accepting it incorrectly).  For example, to handle
\mab\ss\mba\ race in Figure \ref{pat}d, \mr{5358} simply ignores
the unexpectedly-late \mba\sub{finish} message that arrives after
A\sub{AM} sends an \mab\sub{kill} message.
%
Finally, the patch can simply {\em accept} the untimely message
by {\em re-using} existing handlers (as opposed to ignoring it or throwing
an error).  For example, \mr{2995}'s fix changes the node AM to accept
an unexpectedly-early expiration message using an existing
handler that was originally designed to accept the same message at a
later state of AM.  \mr{5198}'s fix handles the atomicity violation by
using an existing handler and simply cancels the atomicity violating
local operation.
%
The rest of the fix-handling cases require new computation
logic to fix bugs.




% -------------------------------------------------
\subsection{Fault/Reboot Timing Bug Fixes}
\label{fix-crash}






Table \ref{tab:fix-crash} summarizes fix strategies for fault/reboot
timing bugs.  Unlike message timing, only rare bugs can be fixed by
controlling the triggering timing either globally or
locally (\eg, by controlling the timing of the fault recovery
actions).  A prime example is an HBase cluster-wide restart scenario
(\hb{3596}).  Here, as A shuts down earlier, B assumes responsibility
of A's regions (via a region-takeover recovery protocol), but soon B
shuts down as well with the regions still locked in ZooKeeper and the
takeover cannot be resumed after restart.  The patch simply adds a
delay before a node starts region takeover so that it will likely get
forced down before the takeover starts.



\begin{table}[t]
\small
\centering
\begin{tabular}{lcccccccc}
\toprule
   & \multicolumn{2}{c}{Fix Timing} && \multicolumn{5}{c}{Fix Handling}\\
\cmidrule{2-3}
\cmidrule{5-9}
   &  G & L &&
   \multicolumn{2}{c}{Detect} &&
   \multicolumn{2}{c}{Recover}\\
\cmidrule{5-6}
\cmidrule{8-9}
   &&&&
   TO & Msg & &
   Canc & Misc \\
\midrule
CA & 1   & 0 && 3 & 2 && 4 & 6   \\
HB & 0   & 1 && 3 & 1 && 6 & 1   \\
MR & 2   & 1 && 1 & 1 && 2 & 1   \\
ZK & 0   & 3 && 0 & 1 && 1 & 7   \\
\midrule
All& 3   & 5 && 7 & 5 && 13& 15  \\
\bottomrule
\end{tabular}
\mycaption{tab:fix-crash}{Fix strategies for fault/reboot 
timing bugs (\sec{\ref{fix-crash}})}
{Some bugs require more than one fix strategy.}
\end{table}


\if 0
\jirafootnote{fix-crash}{tab:fix-crash}{
\spa \ca{2083}, \mr{4832}; % FixFaultTimeGlobal
\spb \hb{3596}, \mr{5476}, \zk{1264}; % FixFaultTimeLocal
\spc \ca{5393}, \hb{6060}, \mr{3186}; % FixFaultHandTO
\spd \ca{2496}, \hb{6317}, \mr{5489}, \zk{1154}; % FixFaultHandMsg
\spe \hb{10090},\mr{3186}, \zk{1154}; % FixFaultHandCS
\spf \ca{2590}, \hb{3446}, \mr{2783}, \zk{1419}; % FixFaultHandOth
}
\fi

For the majority of fault timing bugs, their patches conduct two
tasks: (1) detect the local/global state inconsistency caused by the
fault and (2) repair/recover the inconsistency.  The former is
accomplished through timeouts, additional message
exchanges, or others (omitted from Table \ref{tab:fix-crash}).
The latter can be achieved by simply canceling
operations or adding new computation logic.


\finding{A \textit{small number of fix strategies} have 
fixed most DC bugs.  A few DC bugs are fixed by \textit{disabling} 
the triggering timing (\pctFixTime\ in Figure \ref{bars}p), 
occasionally through extra messages and mostly 
through local operation re-orderings.
Most DC bugs are fixed by better handling the triggering 
timing, most of which do not introduce new computation logic --- 
they \textit{ignore} or \textit{delay} messages,
\textit{re-use} existing handlers, and \textit{cancel} computation 
(\pctFixHandEasy).}
%Figure \ref{bars}p (\BFIX) summarizes the numbers above.}

