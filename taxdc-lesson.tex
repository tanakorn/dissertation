\vten

\section{Lessons Learned}
\label{sec-less}

We now discuss the lessons learned, implications to existing tools
and the opportunities for new research in combating DC bugs.  
Although many high-level
directions can be adopted from work on LC bugs, there are many
interesting challenges and opportunities unique to DC bugs.

% ===========================================================
\subsection{Fault Paths and Multi-Protocol Interactions}
\label{less-fault}


Individual protocols tend to be robust in general.  Only 18 DC bugs
occur in {\em individual} protocols {\em without} any input 
fault condition; only
8 of them are in foreground protocols.  On the other hand, a large
majority of DC bugs happen due to concurrent executions of multiple
protocols and/or different fault timings (Finding {\bf \#2}).
%
This has a tremendous implication to input testing: {\em all types of
  verification, testing, and analysis approaches must consider fault
  injections and multiple protocols as input conditions.}
%
Although recent work has paid attention to this
\cite{Gunawi+11-FateDestini, Joshi+11-PreFail, 
Yuan+14-SimpleTesting}, we emphasize
that all forms of faults (\sec\ref{met-pres}) must be exercised.


% ======================================================
\subsection{Distributed Systems Model Checkers}
\label{less-dmck}

Assuming the necessary input conditions are exercised, the next
question is: can we test different event re-orderings to hit the
triggering timing (\sec\ref{trig-time})?  This is the job of
distributed system model checkers (dmck), which are gaining popularity
recently \cite{Guo+11-Demeter, 
Killian+07-LifeDeathMaceMC,
  Leesatapornwongsa+14-Samc, Simsa+10-Dbug,
  Yang+09-Modist}.  Dmck works by intercepting distributed events and
permuting their ordering.  The more events included, the more
scalability issues will arise due to state-space explosion.
%
To date, {\em no dmck completely controls the timings of
  \underline{all} necessary events} that might contribute to the
triggering timing (Finding {\bf \#1}).  MaceMC
\cite{Killian+07-LifeDeathMaceMC} only reorders messages and network
disconnections.  MoDist \cite{Guo+11-Demeter} exercises timeouts and
Demeter \cite{Guo+11-Demeter} intercepts messages and local
computation but they do not explore different timing of multiple
crashes and reboots.  SAMC \cite{Leesatapornwongsa+14-Samc} exercises
multiple faults but does not support timeout and thread controls.
Also, none of the above include storage faults
or timing issues \cite{Hao+16-TailAtStore}.
%
Therefore, continued research on scalable exploration algorithms is
needed, specifically 
% to reduce the state-space explosion 
when {\em all} the necessary events need to be controlled.
This could be helped by DC bugs' triggering scope characteristics
(Finding {\bf \#3}), just like that in
LC model checkers \cite{madanpldi07}.


% ======================================================
\subsection{Domain-Specific Specifications}
\label{less-spec}

Now, assuming the necessary events are controlled, the next question
is: do we have the specification to judge the manifestation of a bug?  
This is a plague for many tools.  For example,
Demeter does not find new bugs \cite{Guo+11-Demeter} and SAMC
\cite{Leesatapornwongsa+14-Samc} finds two new bugs.
Conversations with the authors suggest that their target systems do
not deploy detailed specifications, and thus some bugs are left
uncaught.  Deploying generic ``textbook'' specifications (\eg, ``only
one leader exists'') does not help as they could lead to false
positives (\eg, ZooKeeper allows two leaders at a single point in
time).  Many research papers on specifications only deploy few of them
\cite{Gunawi+11-FateDestini, Liu+08-D3S, Reynolds+06-Pip}.
Developers also
bemoan the hard-to-debug fail-silent problems \mr{3634} and
prefer to see easier-to-debug fail-stop bugs.  
%

On the positive side, \pctErrExp\ of DC bugs lead to
explicit first errors (Finding {\bf \#4}), implying that sanity checks
already in software can be harnessed as 
specifications
(more in \sec\ref{less-det}). On the other side, compared to
single-machine systems, distributed systems are much more capable
of masking errors. Therefore, these error
specifications have to be used with caution to avoid false 
positives. Furthermore, 
\pctErrImp\ of DC bugs lead to silent first errors (Finding {\bf \#4}).
Many of them proceed to ``silent failures'', such as data loss,
node hangs, etc. Even if they become explicit
errors later, these explicit errors could be far away from the 
initial triggering conditions (\eg, Figure \ref{fig-paxos}).
In short, {\em no matter how sophisticated the tools are, they are
  ineffective without accurate specifications}.
%
This motivates the creation or inference of local specifications
that can show early errors or symptoms of DC bugs.


% ======================================================
\subsection{Bug Detection Tools}
\label{less-det}

%We emphasize that the state of the art solutions for dc bugs fall into
%three camps: testing and model checking \cite{x}, verifiable
%distributed system frameworks \cite{x}, postmortem monitoring and
%debugging \cite{x}.  

We now discuss bug detection tools, which are
unfortunately rare for DC bugs, although very popular for
LC bugs
%
\cite{pacer,
  flanagan09fasttrack, 
  satish.pldi14, 
  avio.asplos06, 
  madanpldi07,
  savage97eraser}.
%
Bug detection tools look for bugs that match specific patterns.  They
cannot provide bug-free proof, but can be efficient in discovering
bugs when guided by the right patterns.
Our study provides guidance and patterns that can be exploited by
future DC bug detection.

\paragraph{Generic detection framework.}

Finding {\bf \#1} implies that
detecting DC bugs, particularly message-timing DC bugs, should focus
on two key tasks: (1) obtaining timing specifications, including order
and atomicity specifications among messages and computation; and (2)
detecting violations to these specifications through dynamic or static
analysis.

\paragraph{Invariant-guided detection.}

Likely program invariants can be learned from program
behaviors, and used as specifications in bug detection
\cite{engler01bugs, daikon00, avio.asplos06}.
The key challenge is to design
simple and suitable 
invariant templates.  For example, ``function $F_1$ should
always follow $F_2$'' is a useful template for API-related semantic
bugs \cite{engler01bugs};
%``variable $v$ should only be accessed by instruction $i_1$ and $i_2$''
%is good for memory bugs 
%\cite{accmon}; 
``the atomicity of accesses $a_1$ and $a_2$ should never be violated''
is effective for LC bugs \cite{avio.asplos06}.
%
%
Finding {\bf \#1} about triggering timing and Finding
\#{\bf 4} about error patterns provide empirical evidence that these
templates can be effective for DC bugs: ``message $bc$ should
arrive at $C$ before message $ac$ ($ca$) arrives (leaves)''; 
``message $ab$
should never arrive in the middle of event $e$ on node $B$''; and
``message $ab$ should always be replied''.

\paragraph{Misconception-guided bug detection.}

Knowing programmers' misconceptions can help bug detectors
focus on specifications likely to be violated.  LC bug
researchers have leveraged misconceptions such as {\it ``two near-by
  reads of the same variable should return the same value''}
\cite{avio.asplos06} and {\it ``a condition checked to be true should
  remain true when used''} \cite{ifcon.hpca14}.
%
%
Finding {\bf \#6} reveals that DC-unique common misconceptions, such
as {\it ``a single hop is faster than double hops''}, 
{\it ``local computation is   faster than one-hop message''}, 
{\it ``atomic blocks cannot be broken''}
can help DC bug detection.


\paragraph{Error-guided bug detection.} 

Finding {\bf \#4} shows that many DC bugs lead to explicit
local/global errors, which implies that timing specifications for many
DC bugs can be inferred backward based on explicit errors.  For
example, program analysis may reveal that a state-machine exception
$e$ will arise whenever $C$ receives message $ac$ before $bc$, which
provides a timing specification ($ac$ arrives before $bc$)
whose violation leads to a {\it local error}; or, the analysis may
reveal that exception $e$ arises whenever node $B$ receives a message
$cb$ from node $C$ and $C$ only sends $cb$ when $ac$ arrives at
$C$ before $bc$, which provides a timing specification whose
violation leads to a {\it wrong-message global error}; and so on.  
\if 0
We
have also observed two specific errors that have each been encountered
by at least 10 DC bugs. 
One is type-state error, such as task state-machine exceptions
in MapReduce, znode state-machine exceptions in ZooKeeper,
etc.; the
other is resource-usage error, such as use before creation or use
after destroy.  Bug detection tools tailored for these errors
can be very accurate.
\fi



\paragraph{Software testing.}

Testing takes a quarter of all software development resources,
and is crucial in exposing bugs before code
release.  Although many testing techniques have been proposed for LC
bugs \cite{madan.asplos10,
ctrigger.asplos09, racefuzzer}, there have been few for
DC bugs \cite{jcute}.
%
%
Finding {\bf \#2} implies that test input design has to consider
faults, concurrent protocols, and background protocols. Finding \#{\bf
  3} implies that {\it pairwise testing}, which targets every pair of
message ordering, every pair of protocol interaction, and so on, will
work much more effectively than {\it all combination testing}, which
exercises all possible total orders and interactions of all
messages and all protocols.
%
For example, a large number of DC bugs (Figure \ref{bars}d-f) can be 
found with inputs of at most two protocols, crashes and reboots.


\vten % good spacing

% ======================================================
\subsection{Failure Diagnosis}
\label{less-diagnose}


Given failure symptoms, distributed systems developers have to reason 
about many nodes to figure out the triggering and root cause of 
a failure.
Our study provides guidance to this challenging process of
failure diagnosis.

\paragraph{Slicing/dependence analysis.} 

Identifying which instructions can affect the outcome of an
instruction $i$ is a widely used debugging technique for deterministic
sequential bugs.  However, it cannot scale to the whole distributed
systems, and hence is rarely used.
%
%
Finding {\bf \#3} indicates that most DC bugs have deterministic error
propagation; Finding {\bf \#4} shows that many DC bugs have their
errors propagate through missing or
wrong messages.  Therefore, per-node dependence analysis that can
quickly identify whether the generation of a local error depends on
any incoming messages would help DC bug failure diagnosis to get
closer and closer to where the triggering events happen.




\paragraph{Error logging.}  

Error logging is crucial in failure diagnosis. If the
first error of a DC bug is an explicit local error, the error log can
help developers quickly identify the triggering node and focus their
diagnosis on one node.
%
%
Finding {\bf \#4} unfortunately shows that only \pctErrLocExp\ of DC bugs
lead to explicit local errors. This finding motivates future
tool to help make more DC bugs lead to explicit local errors.


\paragraph{Statistical debugging.}

Comparing success-run traces with failure-run traces can help identify
failure predictors for semantic bugs \cite{liblit03} 
and concurrency bugs \cite{cci.oopsla10}
in single-machine software.  The key
design question is what type of program properties should be
compared between failure and success runs. For example, branch
outcomes are compared for diagnosing semantic bugs but not for LC bugs.
%
%
Finding {\bf \#1} and \#{\bf 3} about triggering timing conditions
provide guidance for applying this approach for DC bugs. We can
collect all message sending/arrival time at runtime, and then find
rare event orderings that lead to failures by contrasting them with
common ``healthy'' orderings (\eg, Figure \ref{pat}b happens 99.99\%
of the time while Figure \ref{pat}a happens 0.01\% of the time).
%
%
Of course, there are challenges. Finding \#{\bf 2} and \#{\bf 3}
show that many DC bugs come from the interactions of
many protocols. Thus, it is not sufficient to only log a chain of messages
originated from the same request, a common practice in request logging
\cite{Chow+14-Mysterymachine}.  Furthermore, some DC bugs are triggered by
message-computation ordering. Therefore, logging messages alone is
not sufficient.

\paragraph{Record and Replay.}

Debugging LC concurrency bugs with record and deterministic replay is
a popular approach \cite{quickrec.isca13,
  doubleplay.asplos11}.  However, such an approach has not permeated
practices in distributed systems debugging.  A ZooKeeper developer
pointed us to a fresh DC bug that causes a whole-cluster outage but
has not been fixed for months because the deployment logs do not
record enough information to replay the bug (\zk{2172}).  There has
been 9 back-and-forth log changes and attachments with 72 discussion
comments between the bug submitter and the developers.  More studies
are needed to understand the gap between record-replay challenges in
practice and the current state of the art \cite{Geels+06-Liblog,
  Liu+07-WiDS}.

\if 0
There needs to be more work in record and replay debugging for DC bugs
beyond the current state of the art which typically focuses on single
protocols The challenges are similar to the ones for statistical
debugging discussed above.
\fi

%Geels+07-Friday, 





\vten % good spacing

% ======================================================
\subsection{Failure Prevention and Fixing}
\label{less-runtime}


\paragraph{Runtime Prevention.}

The manifestation of 
concurrency bugs can sometimes be prevented by
injecting delays at runtime.  This technique has been successfully
deployed to prevent LC bugs based on their timing
conditions \cite{DeadlockImmunity, avisio,conair.asplos13}.
%
%
Finding {\bf \#1} shows that many DC bugs are triggered by untimely
messages and hence can potentially be prevented this way.  For
example, none of the bugs shown in Figure \ref{pat}a--h would happen
if we delay a message arrival/sending or local computation.  Of
course, different from LC bugs, some of these delays have to rely on a
network interposition layer; similar with LC bugs, some
delays may lead to hangs, and hence cannot be adopted.



\paragraph{Bug Fixing.}

Recent work automatically fixes LC bugs by inserting lock/unlock or
signal/wait to prohibit buggy timing \cite{cfix.osdi12,
  grail.fse14, wang.osdi08}.
%
%
Finding {\bf \#5} shows that the same philosophy is promising for
\pctFixTime\ of studied DC bugs. Our study 
shows that this approach has to be tweaked to focus on using global
messages (\eg, ACKs) or local operation re-ordering, instead of lock
or signal, to fix DC bugs.  Finding {\bf \#5} indicates that
\pctFixHandEasy\ of those DC bugs are fixed by shifting message handlers,
ignoring messages, and canceling computation, without adding new
computation logic.  This presents a unique opportunity for developing
new and more aggressive fixing techniques.


\vfive  % good spacing

% ===========================================================
\subsection{Distributed Transactions}
\label{less-tx}

\if 0
\jirafootnotable{less-tx}{
\spa \hb{4729}, \hb{4540}, \hb{4539}, \hb{4015}.
}
\fi



\begin{figure}
\fbox{
\begin{minipage}{\textwidth}
\vspace{10pt}
\begin{quote}
%---------------------------------
{\bf \hb{9095}:}
\enumerate{
\item RS successfully OPENED region R,
\item RS notifies ZK that region R is OPENED,
\item ZK continues region R state msg to Master,
\item Master starts processing OPENED msg,
\item Meanwhile RS CLOSED region R (asked by Client),
\item RS notifies ZK that region R is CLOSED,
\item Master asks ZK to delete znode for region R,
\fev{concurrently racing with step 6!},
\item ZK deletes region R's znode,
\item Master never assigns region R to any RS.
\fev{R becomes an orphan!}
}
%---------------------------------
\end{quote}
\vspace{10pt}
\end{minipage}
}
\mycaption{fig-hbase}{Race of HBase's messages to ZooKeeper}{}
%\vten
\end{figure}



In the middle of our study, we ask ourselves: if DC bugs can be
theoretically solved by distributed transactions, why doesn't such
technique eliminate DC bugs in practice?  Our answers are:
%
first, the {\em actual implementations of theoretically-proven
  distributed transactions are not always correct}
(as also alluded in other work \cite{Burrows06-Chubby, Ongaro+14-Raft}).
For example, new
DC bugs continue to surface in complex distributed transactions such
as ZooKeeper's ZAB and Cassandra's Paxos as they are continuously
modified.
%
Second, {\em distributed transactions are only a subset of a full
  complete system}.  A prime example is the use of ZooKeeper in HBase
for coordinating and sharing states between HBase masters and region
servers.  Although ZooKeeper provides linearization of updates, HBase
must handle its concurrent operations to ZooKeeper,
for example, step 6 and 7 in Figure \ref{fig-hbase};
there are many other similar examples.%\spa.
%
Put simply, there are many protocols that do not use distributed
transactions, instead they use domain-specific finite state machines,
which should be tested more heavily.


Another approach to eliminate non-deterministic bugs in distributed
protocols is by building deterministic distributed systems.  However, the
technique is still in its infancy, at least in terms of the impact to
performance (\eg, an order of magnitude of overhead \cite{Hunt+13-DDOS}).




% ======================================================
\subsection{Verifiable Frameworks}
\label{less-others}

Recently there is a growing work on new programming language frameworks
for building verifiable distributed systems \cite{Desai+13-PLang,
Hawblitzel+15-IronFleet, Wilcox+15-Verdi}, but they typically focus on the
main protocols  and not the full system including
the background protocols.  One major challenge is that 
just for the basic read and write protocols,
the length of the
proofs can reach thousands of lines of code, potentially
larger than the protocol implementation.
Unfortunately, our study shows that the complex
interaction between foreground and background protocols can lead to DC
bugs.  Therefore, for complete real-world systems, verification of the
entire set of the protocols is needed.
%

\subsection{LC bugs vs. DC bugs}
\label{less-lcdc}
There are clearly similarities between LC bugs and DC bugs, as, 
by definition, they are both timing-related non-deterministic bugs.
Many DC bugs contain LC components:
untimely messages may lead to unsynchronized accesses from 
multiple threads or multiple event-handlers 
\cite{vechev.oopsla13,satish.pldi14} in a single machine.
It is probably not a surprise that atomicity violations
and order violations are two dominant triggering timing conditions
for both LC and DC bugs (Finding {\bf \#1}). 
Our observation of the small triggering scope of most DC bugs
(Finding {\bf \#3})
is similar with that for LC bugs, which may be related to the nature
of the bug sets ---
more complicated bugs may be more difficult to fix, and
hence less likely to be included in empirical studies. 

There are also many differences between LC bugs and DC bugs, as they originate
from different programming paradigms and execution
environments. For example, order violations are much more common in DC bugs
than those in LC bugs (Finding {\bf \#1}); faults and reboots are much 
more common in DC bugs than those in LC bugs (Finding {\bf \#2}); 
the diagnosis of many DC bugs will have to reason beyond one node, 
clearly different from that of LC bugs (Finding {\bf \#4});
the fix
strategies for DC bugs are very different from those of LC bugs, because
enforcing global synchronization is difficult (Finding {\bf \#5}).
