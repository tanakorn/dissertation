
\section{DMCK Framework and Terms}
\label{mot-bgterms}




As mentioned before, we define {\em dmck} as  software model checker
that checks distributed systems directly at the implementation level.
Figure~\ref{fig-dmck} illustrates a dmck integration to a target
distributed system, a simple representation of existing dmck
frameworks~\cite{Guo+11-Demeter, Killian+07-LifeDeathMaceMC,
  Simsa+10-Dbug, Yang+09-Modist}.  The dmck inserts an interposition
layer in each node of the target system with the purpose of
controlling all important events (\eg, network messages, timeouts) and
preventing the target system to process the events until the dmck
enables them.  A main dmck mechanism is the permutation of events; the
goal is to push the target system into all possible ordering scenarios.
For example, the dmck can enforce \ts{abcd} ordering in one execution,
\ts{bcad} in another, and so on.



% List all the terms
We now provide an overview of basic dmck terms we use in this thesis
and Figure~\ref{fig-dmck}.
%
Each node of the target system has a {\em local state} (\ls),
containing many variables.  An {\em abstract local state} (\als) is a
subset of the local state; dmck decides which \als\ is important to
check.
%
The collection of all (and abstract) local states is the {\em global
  state} (\gs) and the {\em abstract global state} (\ags)
respectively.  
%
The {\em network state} describes all the {\em outstanding messages}
currently intercepted by dmck (\eg, \ts{abd}).
%
To model check a specific protocol, dmck starts a {\em workload
  driver} (which restarts the whole system, runs specific workloads,
\etc).  Then, dmck generates many (typically hundreds/thousands)
executions; an {\em execution} (or a {\em path}) is a specific
ordering of events that dmck enables (\eg, \ts{abcd}, \ts{dbca}) from
an initial state to a termination point.
%
A {\em sub-path} is a subset of a path/execution.
%
An {\em event} is an action by the target system that is intercepted
by dmck (\eg, a network message) or an action that dmck can inject
(\eg, a crash/reboot).
%
Dmck enables one event at a time (\eg, \ts{enable(c)}).
%
To permute events, dmck runs {\em exploration methods} such as
brute-force (\eg, depth first search) or random.  
%
%
As events are permuted, the target system enters hard-to-reach
states.  Dmck continuously runs state {\em checks} (\eg, safety 
checks) to verify the system's correctness.
%
To reduce the state-space explosion problem, dmck can employ {\em
  reduction policies} (\eg, DPOR or symmetry).  A policy is {\em
  systematic} if it does not use randomness or bug-specific knowledge.
%
In this work, we focus on advancing systematic reduction policies.

