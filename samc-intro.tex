%State of the art does not help.

As we discuss in Section \ref{sec-bg-dmck}, a recent popular approach to unearth
DC bugs is adopting distributed system model cheker or dmck. However, due to the
complexity of DC bugs that we discuss in Chapter \ref{chp-taxdc} (\eg,
interactions of multiple protocols, and various and multiple faults), the state
of the art of dmcks still cannot work effectively \xxx{cite Modist, dbug,
demeter, macemc}. We observe that the existing systematic reduction policies
cannot find bugs quickly, and cannot scale with the inclusion of fault events.

In this chapter, we discuss how to advance the state of the art by leveraging
semantic awareness to assist model checking, and introduce ``Semantic-Aware
Model Checking'' (SAMC), a white-box model checking approach that takes semantic
knowledge of how events (\eg, messages, crashes, and reboots) are processed by
the target system and incorporates that information in reduction policies. We
first discuss background of dmck and the state of the art in Section \ref{sec-mot},
then we discuss about the concept of SAMC and its implementation in Section
\ref{sam-samc} and \ref{sec-impl} respectively. We evalute SAMC by comparing with the
state of the art in Section \ref{sec-eval}, and lastly, we discuss other issues in
adopting SAMC to check distirbuted systems in Section \ref{discuss}.

%To show the intuition
%behind SAMC, we first give an example of a simple leader election protocol.
%Then, we present SAMC architecture and our four reduction policies.

