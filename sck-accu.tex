




\subsection{Accuracy}
\label{eval-accu}

\def \fff        {$f$}
\def \flaps      {\textit{\#flaps}\xspace}
\def \gosLast    {$T_{lastGossip}$\xspace}
\def \gosAvg     {$T_{avgGossip}$\xspace}
\def \gosProc    {$T_{gossipExec}$\xspace}
\def \supProc    {$T_{stateUpdate}$\xspace}
\def \hops       {\textit{\#hops}\xspace}

\def \ringTable  {$Size_{ringTable}$\xspace}
\def \newStates  {$Size_{newStates}$\xspace}
\def \cpuSpeed   {$CPU$\xspace}


Next, we provide a detailed accuracy evaluation of \sck.  Due to space
constraints, this section only focuses on one bug (\caone \cite{CA-One}) 
while the next
section briefly discuss other bugs we reproduced.




% metrics
Figure \ref{fig-form}a-d presents the  internal metrics within
Cassandra failure detection protocol that we measured for {\em every pair}
of nodes.  That is, the algorithm runs on every node A for every peer B.
%
Figure \ref{fig-accu}a-d compare in detail the accuracy of \sck compared
to real deployments.
%
For example, $x$$=$512 implies the comparison of 512-node colocation in
\sck versus a real deployment of 512 nodes.
%
Note that for \caone, we only need time profiling with offline sampling
(\sec\ref{sc-pil-4}) and no pre-memoized data (\sec\ref{sc-pil-3}).  
%
We use pre-memoization to reproduce Riak \riakone (next section).


% ------------------------ a
Figure \ref{fig-accu}a shows the total number of flaps (alive-to-dead
transitions) observed in the whole cluster during bootstrapping.  
As shown, \sck closely mimics
real deployment scenarios.  Most importantly here, a significant \flaps
does not appear until 256-node deployment, 
hence mini-cluster extrapolation techniques will not work (\sec\ref{mot-state}).
%
Figure \ref{fig-form}a defines that \flaps depends on 
\phi \cite{Hayashibara+04-PhiFailureDetector}.  Every node A
maintains a \phi value for a peer node B (a total of $N$$\times$$(N$$-$$1)$
variables to monitor).  If \phi$>$8 for B, A will declare B dead (a flap).

% ------------------------ b
Figure \ref{fig-accu}b shows the maximum \phi values observed for every
peer node.  For example, for the 512-node setup, the whisker plots show the
distribution of the maximum \phi values observed for each of the 512
nodes.  As shown, the larger the cluster, more \phi values exceeds the
threshold value of 8, hence the flapping.
%
Figure \ref{fig-form}b points that \phi depends on the average
inter-arrival time of when new gossips about B arrives at A (\gosAvg) and the
time since A heard the last gossip about B (\gosLast); the ``last gossip''
is the last version number received (\sec\ref{mot-bug}).  The point is that
\gosLast should not be much higher than \gosAvg.




% d) $T_{avghbperiod}$ = h(all previous $T_{hbsilence}$) \\



\def \fgap {~~~~~~~~~~}
\def \fgap {~~~~}


\begin{figure}

\small
\centering


\begin{spacing}{1.5}
\begin{tabular}{|p{3.2in}|}
\hline

a) \flaps = \fff $($ \phi $>$ $8$ $)$ \\

b) \phi = \fff $($ \gosAvg, \gosLast $)$ \\
\fgap   \gosAvg = avg. of last 1000 \gosLast \\

c) \gosLast = \fff $($ \hops, \gosProc $)$ \\
\fgap \hops = $log(N)$ on average \\
\fgap \gosProc = \supProc (if new state changes) \\

d) \supProc = \fff $($ \ringTable, \newStates $)$ \\

\fgap   \ringTable $\leq$ $N$$\times$$P$ and \newStates $\leq$ $N$ \\



\hline
\end{tabular}
\end{spacing}

\vfive % orphan : we have extra space on page 10

\mycaption{fig-form}{Cassandra internal metrics (\sec\ref{eval-accu})}{Above
are the metrics we measured within the Cassandra bootstrap protocol
for measuring \sck accuracy (Figure \ref{fig-accu}). 
``f'' represents ``a function of'' (\ie, an arbitrary function).}


\end{figure}





% ------------------------ c
Figure \ref{fig-accu}c shows the whisker plots of gossip inter-arrival
times (\gosLast) that we collected for every A-B pair.  For example, for the
512-node setup, the whisker plots represent the distribution of 
around 41 million 
gossip inter-arrival times; this large number is because a message
contains gossips of many peer nodes.  The figure shows that in larger
clusters, new gossips do not arrive as fast as in smaller clusters,
especially at high percentiles.
%
Figure \ref{fig-form}c shows that \gosLast depends on how far B's new
gossips propagate through other nodes to A (\hops) and the gossip
processing time in each hop (\gosProc).  The \hops is stable
at $log(N)$ on average in \sck and real deployment (not shown).  The
latter (\gosProc) is essentially state-update processing time (\supProc)
whenever there are state changes, which is the culprit.

% ------------------------ d
Figure \ref{fig-accu}d ({\em in log scale}) shows the whisker plots of
state-update processing time (\supProc); in the 512-node setup, we
measured around 25,000 state-update invocations.  The figure shows that at high
percentiles, \supProc is scale-dependent.  As explained in 
Figure \ref{fig-form}d (and Section \ref{mot-bug}), \supProc complicatedly
depends on a scale-dependent 2-dimensional input (\ringTable and
\newStates); a node's \ringTable depends on how many nodes it knows,
including the partition arrangement ($\leq$$N$$\times$$P$) and \newStates
($\leq$$N$) increases as cluster size increases.
%
%The bug was not expected because the median of \supProc is not
%scale-dependent.
%
Note that the \supProc in \sck comes from the sampling-based time
profiling (\sec\ref{sc-pil-4}), which is relatively accurate as the figure
shows.




\def \fgw {2.7in}



\begin{figure}[t]

%\centerline{

%}



\centerline{
%\hmina
\includegraphics[width=\fgw]{F/accu/eps/flap}
%\hminb
\includegraphics[width=\fgw]{F/accu/eps/phi}
%\hmina
%\includegraphics[width=\fgw]{F/accu/eps/hb}
%\hminb
%\includegraphics[width=\fgw]{F/accu/eps/proc-log}
}

%\vfive % orphan : we have extra space on page 10

\centerline{
%\hmina
\includegraphics[width=\fgw]{F/accu/eps/hb}
%\hminb
\includegraphics[width=\fgw]{F/accu/eps/proc-log}
}

%\centerline{
%\includegraphics[width=\fgw]{F/accu/eps/proc}
%\includegraphics[width=\fgw]{F/accu/eps/hop}
%\includegraphics[width=\fgw]{F/accu/eps/stable}
%}

\vminten
\mycaption[Accuracy in reproducing \caone]{fig-accu}{Accuracy in reproducing \caone}{The 
figures represent the metrics presented in Figure \ref{fig-form},
measured in real deployment (``Real'') and \sck with
different cluster sizes (32, 64, 128, 256, and 512).
Figure title represents the y-axis.}
\vminfive
\end{figure}




\if 0

\hsg{ b - 512
 c - 41295431
 d - 24335
recap, exampe, how many data points.
}

% \includegraphics[width=\fgw]{F/graphs/ca-6127/tsilent_99}
% \includegraphics[width=\fgw]{F/graphs/ca-6127/avgtsilent_99}
\includegraphics[width=\fgw]{F/graphs/ca-6127/avgtsilent}
\includegraphics[width=\fgw]{F/graphs/ca-6127/hop}
\includegraphics[width=\fgw]{F/graphs/ca-6127/hop}
\includegraphics[width=\fgw]{F/graphs/ca-6127/proctime}
\includegraphics[width=\fgw]{F/graphs/ca-6127/proctime_99}
\includegraphics[width=\fgw]{F/graphs/ca-6127/commit}
\includegraphics[width=\fgw]{F/graphs/ca-6127/commit_99}
\includegraphics[width=\fgw]{F/graphs/ca-6127/profiling}
\includegraphics[width=\fgw]{F/graphs/ca-6127/table}
\includegraphics[width=\fgw]{F/graphs/ca-6127/table_med}
\fi



We conclude that \sck mimics similar behaviors as in real deployments and
is accurate for reproducing scalability bugs.
%
As an additional note, we have applied the bug patch in both \sck and real
deployment modes;  Figure \ref{fig-accu}a shows \flaps is always zero in
both modes.



