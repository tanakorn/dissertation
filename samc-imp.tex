\section{Implementation and Integration}
\label{sec-samc-impl}

In this section, we first describe our SAMC prototype, \sampro, which we
built from scratch because existing dmcks are either
proprietary~\cite{Yang+09-Modist} or only work on restricted high-level
languages (\eg, Mace~\cite{Killian+07-LifeDeathMaceMC}).  We will then describe
\sampro\ integration to three widely popular cloud systems,
ZooKeeper~\cite{Hunt+10-ZooKeeperPaper}, Hadoop/Yarn~\cite{Kumar+13-Yarn},
and Cassandra~\cite{Lakshman+09-Cassandra}.  Prior to \sampro, there was no
available dmck for these systems; they are still tested via unit tests, and
the test code size is bigger than the main code, but the tests are far from
reaching deep bugs.


\subsection{\sampro}
\label{imp-pro}

\sampro\ is written in \numLinesSamPro\ lines of code in Java, which
includes all the components mentioned in Section~\ref{mot-bgterms} and
Figure~\ref{fig-dmck}.  The detailed anatomy of dmck has been
thoroughly explained in literature~\cite{Guerraoui+11-McNoNetwork,
  Guo+11-Demeter, Killian+07-LifeDeathMaceMC, Simsa+10-Dbug,
  Yang+09-Modist}, and therefore for brevity, we will not discuss many
engineering details.  We will focus on SAMC-related parts.

% access source code
We design \sampro\ to be highly portable; we do not modify the target code
base significantly as we leverage a mature interposition technology,
AspectJ, for interposing network messages and timeouts.
% local state
Our interposition layer also sends local state information to the
\sampro\ server.
% crashes and reboots
\sampro\ is also equipped with crash and reboot scripts specific to the
target systems.  The tester can specify a budget of the maximum number of
crashes and reboots to inject per execution.
% summ
\sampro\ employs basic reduction mechanisms and advanced reduction policies
as described before.
% checks
We deploy safety checks at the server (\eg, no two leaders).  If a
check is violated, the trace that led to the bug is reported and 
can be deterministically replayed in \sampro.
% other supports
Overall, we have built all the necessary features to show the case of
SAMC.  Other features such as intra-node thread
interleavings~\cite{Guo+11-Demeter}, scale-out
parallelism~\cite{Simsa+12-ScalablePOR}, and virtual clock for network
delay~\cite{Yang+09-Modist} can be integrated to \sampro\ as well.


\vten % orphan text

\subsection{Integration to Target Systems}
\label{imp-targets}


In our work, the target systems are ZooKeeper, Hadoop 2.0/Yarn, and
Cassandra.  ZooKeeper~\cite{Hunt+10-ZooKeeperPaper} is a distributed
synchronization service acting as a backbone of many distributed systems
such as HBase and High-Availability HDFS.  Hadoop
2.0/Yarn~\cite{Kumar+13-Yarn} is the current generation of Hadoop that
separates cluster management and processing components.
Cassandra~\cite{Lakshman+09-Cassandra} is a distributed key-value store
derived from Amazon Dynamo~\cite{DeCandia+07-Dynamo}.

In total, we have model checked \numProtocols\ protocols: ZooKeeper
leader election (ZLE) and atomic broadcast (ZAB), Hadoop cluster
management (CM) and speculative execution (SE), and Cassandra
read/write (RW), hinted handoff (HH) and gossiper (GS).  These
protocols are highly asynchronous and thus susceptible to message
re-orderings and failures.

Table~\ref{tab-policies} shows a real sample of protocol-specific
rules that we wrote.  Rules are in general very short; we only wrote
\numLinesRule\ lines/protocol on average.  This shows the simplicity
of SAMC's integration to a wide variety of distributed system protocols.




\begin{sidewaystable*}[t]
\begin{center}
{\small
%---------------------------------
\begin{tabular}{p{1.9in}|p{2in}|p{2.1in}|p{2in}} 


\multicolumn{1}{c|}{{\bf Local-Message}} &
\multicolumn{1}{c|}{{\bf Crash-Message}} &
\multicolumn{1}{c|}{{\bf Crash Recovery}} &
\multicolumn{1}{c}{{\bf Reboot Synchronization}}
\\

\multicolumn{1}{c|}{{\bf Independence (LMI)}} &
\multicolumn{1}{c|}{{\bf Independence (CMI)}} &
\multicolumn{1}{c|}{{\bf Symmetry (CRS)}} &
\multicolumn{1}{c}{{\bf Symmetry (RSS)}}
\\


\hline  % =====================================================

% ----------------------------------------------- ZLE, LMI (1)

\vminten
{\footnotesize
\begin{alltt}
bool pd : !newVote(m, s);

bool pm : newVote(m, s);

bool newVote(m, s) : {
 if (m.ep > s.ep) 
   return true; 
 else if (m.ep == s.ep)
  if (m.tx > s.tx) 
   return true;
  else if (m.tx == s.tx &&
           m.lid > s.lid) 
   return true;
}
\end{alltt}
}

& % ----------------------------------------------- ZLE, CMI (1)

\vminten
{\footnotesize
\begin{alltt}
bool pg (s, X) : 
 if (s.rl == F && X.rl == L)
  return true;
 if (s.rl == L && X.rl == F
     && !quorumAfterX(s)
  return true;
 if (s.rl == S && X.rl == S) 
  return true;

bool pl (s, X) :
 if (s.rl == L && X.rl == F 
     && quorumAfterX(s)) 
  return true;

bool quorumAfterX(s) :
  ret ((s.fol-1) >= 
        s.all/2);
\end{alltt}
}

& % ----------------------------------------------- ZLE, CRS (1)

\vminten
{\footnotesize
\begin{alltt}
bool pr1(s,C):
 if (s.rl == L && C.rl == F
     && quorumAfterX(s))
  return true;
rals1:\{rl,fol,all\};

bool pr2(s,C):
 if (s.rl == L && C.rl == F 
     && !quorumAfterX(s))
 return true;
rals2: \{rl,fol,lid,ep,tx,clk\}

bool pr3(s,C):
 if (s.rl == F && c.rl == L)
  return true;
rals3: \{rl,fol,lid,ep,tx,clk\}

bool pr4:
 if (s.rl == S)
  return true;
rals4: \{rl,lid,ep,tx,clk\}
\end{alltt}
}


& % ----------------------------------------------- ZLE, RSS (1)

\vminten
{\footnotesize
\begin{alltt}
bool ps1(s,R):
 if (s.rl == L)
  return true;
sals1: \{rl,lid,ep,tx,clk\}

bool ps2(s,R):
 if (s.rl == F)
  return true;
sals2: \{rl,lid,ep,tx,clk\}

bool ps3(s,R):
 if (s.rl == S && 
     s.clk > R.clk)
  return true;
sals3: \{rl,lid,ep,tx,clk\}

bool ps4(s,R):
 if (s.rl == S && 
     moreUpdated(s, R))
  return true;
sals4: \{rl,lid,ep,tx,clk\}

bool moreUpdated(s, R):
 if (R.ep > s.ep)
  return true;
 else if (R.ep == s.ep)
  if (R.tx > s.tx) 
   return true;
  else if (R.tx == s.tx)
   if (R.lid > s.lid)
    return true;
\end{alltt}
}

\end{tabular}
}
%---------------------------------
\end{center}
%
\vminfive
\mycaption[Protocol-Specific Reduction Rules for ZLE]{tab-policies}{Protocol-Specific Reduction Rules for ZLE}{
%
The code above shows the actual protocol-specific rules for
ZLE protocol.  These rules are the inputs to the four reduction policies.
%
Many variables are abbreviated (ep: epoch, tx: latest
transaction ID, lid: leader ID, rl: role, fol: follower count, all: total
node count, clk: logical clock, L: leading, F: following, S: searching,
X/C: crashing node, R: rebooting node). LMI \pc\ and \pi\ predicates are not 
used for ZLE, but used for other protocols. 
%
}
%\vminfive
\end{sidewaystable*}


\if 0
zle-specific rule = 49
zab-specific rule = 33
mapreduce: 35 ..
protocol average = 
\fi


