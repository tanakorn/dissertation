

\section{Discussions}
\label{sec-discuss}

We now discuss the limitations and other aspects of \sck.


% ------------------------------------
{\em What are the limitations of \sck?}
%
Our work focuses on scale-dependent CPU/processing time
(\sec\ref{sec-mot}a).
%
However, there are other scaling problems that lead to I/O and memory
contentions \cite{ Gunawi+14-Cbs, Ousterhout+15-MakingSense,
  Konstantin+10-HDFSScalability}, usually caused by the scale of load
\cite{Bodik+10-WorkloadSpikes, Guo+13-CureIsWorse} or data size
\cite{Nguyen+16-Yak}.
% Armbrust+11-Piql, 
\sck cannot reproduce such issues on one machine as we do not address
memory and IO emulation.
%
We are only aware of one work \cite{Wang+14-Exalt} that scale-tests
implementation code with Big-Data workloads (more in
\sec\ref{sec-rel}).
%
We believe there are many open problems to solve in this new research
area.


% -----------------------------------
{\em Can \sck reach $>$1000-node colocation factor?}  So far, \sck is
limited by the single machine's resources.  To increase colocation factor,
a higher-end machine can be used.   Another approach is to 
extend \sck\ to run on multiple machines.
However, this means that we need to enable back the networking library,
which originally already caused a colocation 
bottleneck (\sec\ref{sc-mfr}).  Techniques such as automated
batching might help.  We leave this challenge for future work.



% ------------------------------------
{\em Is \sfind sufficient (without \stest) to reveal scalability bugs?}
Building a program analysis that covers all paths and understands the
cascading impacts without false positives is challenging.  Not all
scale-dependent loops imply buggy code.  For example, in \caone, if
Cassandra processes gossips in a multi-threaded manner, the long
processing time might not cascade to failures.


% ------------------------------------
{\em Can \stestp run without \stestm?}
%
Ideally, the fast \stestp (replay with PIL) should run without the long
\stestm (pre-memoization phase), possible only if we can construct all
input-output pairs.  However, in the context of large-scale,
non-deterministic systems, constructing all possible pairs requires an
{\em ``infinite''} time and storage space, because node states (the input)
typically depend on the {\em order} in which messages arrive (which tends
to be random).
%
Let us consider Riak's rebalance protocol, where each node gossips
randomly its partition table, until a globally balanced state is reached.
With $N$$=$256 and $P$$=$64 (\#key-partitions$/$node), we counted 2489
rebalance iterations with a set of specific input-output pairs in one
complete run.  In another run, a different set of input-output pairs is
produced due to gossip randomness.  Our calculation suggests $(N^{NP})^2$
(``infinite'') possible pairs.
%
In contrast, \stestm only needs to store around 2500 input-output pairs
(1.3 GB of memoized data).


% ------------------------------------
{\em Can function-level performance profiling hints of scalability bugs?}
%
It is possible, but in our experience, using average performance numbers
do not help.  Figure~\ref{fig-accu}d shows that gossip/update processing
time can wildly range from 0.001 to 4 seconds.  The wide range is due
the many \ts{if-else} conditions that depend on the current state.


% ------------------------------------
{\em What do developers think of \sck?}  We have ongoing communications
with Cassandra and HDFS developers, who initially prefer an approach
without source code modification, as expected.  We tried
(\sec\ref{sc-test}) but again, current distributed systems, OS, and
language runtime do not support a naive-packing approach.  This is another
open research problem to address.
%
Nevertheless and most importantly, the developers agree that debugging
bugs in large-scale deployments is onerous and not economical and having a
testing framework such as \sck would be very valuable.
%
Currently, a Hadoop/HDFS PMC\footnote{PMC: Project Management Committee} member
and committer is collaborating with us to find more scalability bugs in HDFS.

%  (by covering more scale-dependent data structures).


% ------------------------------------
{\em Can \sck run on multiple machines?}  The concept of distributed \sck
can potentially increase the testing scale (and might also raise new
challenges), as we already discussed above.


