

\begin{figure}[t]

{\small
\begin{alltt}
broadcast() { sendMsgToAll(role, leaderId); }
quorumOkay() { return (follower > nodes / 2); }

// pr1
if (role == L && C.role == F && quorumOkay())
  follower--;

// pr2
if (role == L && C.role == F && !quorumOkay())
  follower = 0;
  role = S;
  broadcast();

// pr3
if (role == F && C.role == L) 
  leaderId = myId;
  broadcast();
\end{alltt}
}
%\vminten 
\mycaption[Crash Recovery in Leader Election]{code-crash}{Crash Recovery in Leader Election}{
  The figure shows a simplified example of crash recovery in a leader
  election protocol.  The code runs in every node.  \ts{C} implies the
  crashing node; each node typically has a view of the states of its
  peers.  Three predicate-recovery pairs are shown (\prone, \prtwo,
  and \prtri).  In the first, if quorum still exists, the leader
  simply decrements the follower count.  In the second, if quorum
  breaks, the leader falls back to searching mode (\ts{S}).  In the
  third, if the leader crashes, the node (as a follower) votes for
  itself and broadcasts the vote to elect a new leader. }
%\vmin
\end{figure}
