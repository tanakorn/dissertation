

\section{Methodology}
\label{sec-met}

%\myquote{``I'm suspecting there is a race condition.'' --- \zk{1496}}


% -------------------------------------------------
\vfifteen
\subsection{Basic Definitions}
\label{met-def}

A {\em distributed concurrency (DC) bug} is a 
concurrency bug in distributed
systems caused by distributed events that can occur in
non-deterministic order.  An {\em event} can be a message arrival/sending, 
local computation, fault, and reboot.
%
A {\em local concurrency (LC) bug} is a 
concurrency bug that happens locally
within a node due to thread interleaving.
%
In our model, a {\em distributed system} is a collection of
shared-nothing nodes.  Each node can run multiple protocols in
multiple threads.
%


% -------------------------------------------------
\subsection{Target Systems and Dataset}
\label{met-data}


Our study examined bugs from four widely-deployed
open-source datacenter distributed
systems that represent a diverse set of system architectures: Hadoop
MapReduce (including Yarn) \cite{HadoopWeb} representing distributed
computing frameworks, HBase \cite{HBaseWeb} and Cassandra
\cite{CassandraWeb} representing distributed key-value stores (also
known as NoSQL systems), and ZooKeeper \cite{ZooKeeperWeb}
representing synchronization services.
%
They are all fully complete systems containing many complex concurrent
protocols.  Throughout the chapter, we will present short examples of DC
bugs in these systems.  Some detailed examples are illustrated in Figure
\ref{fig-zook}, \ref{fig-paxos} and \ref{fig-hbase}.

The development projects of our target systems are all hosted under
Apache Software Foundation wherein organized issue repositories (named
``JIRA'') are maintained.  To date, across the four systems, there are
over 30,000 issues submitted.  One major challenge is that issues
pertaining to DC bugs do not always contain plain terms such as
``concurrency'', ``race'', ``atomicity'', \etc\ Scanning all the
issues is a daunting task.  Thus, we started our study from an open
source cloud bug study (CBS) database \cite{CBSWeb}, which already
labels issues related to concurrency bugs.  However, beyond simple
labeling, the CBS work did not differentiate DC from LC bugs and did
not dissect DC bugs further.

From CBS, we first filtered out LC bugs, then exclude 
DC bugs that do not contain clear description, and finally
randomly picked \numDcBugs\ samples
from the remaining detailed DC bugs, specifically \numDcCA\
Cassandra, \numDcHB\ HBase, \numDcMR\ Hadoop MapReduce, and \numDcZK\
ZooKeeper DC bugs, reported in January 2011-2014 (the time range of
CBS work).  
We have seen much fewer clearly explained DC bugs in CBS from 
Cassandra and ZooKeeper than those from HBase and Hadoop MapReduce, which 
may be related to the fact that they are different types of distributed
systems.  For example,  ZooKeeper, as a
synchronization service, is quite robust as it is built on the
assumption of event asynchrony since day one. Cassandra was built on
eventual consistency, and thus did not have many complex transactions,
until recently when Cassandra adopts Paxos.  We still see new DC bugs
throughout 2014-2015 (some pointed to us by the developers); they can
be included into \tdc\ in the future.




% ---------------------------------
\subsection{Taxonomy}
\label{met-tax}


\definecolor{Gray}{gray}{0.75}
\begin{table}[!htb]
%\footnotesize
\centering
\begin{tabular}{lp{5.5in}}
\toprule
%\rowcolor{Gray}
\multicolumn{2}{c}{{\bf Triggering} }\\
\midrule
\multicolumn{2}{l}{\it What is the triggering timing condition?}\\
&{Message arrives unexpectedly late/early}\\
&{Message arrives unexpectedly in the middle}\\
&{Fault (component failures) at an unexpected state}\\
&{Reboot at an unexpected state}\\
\multicolumn{2}{l}{\it What are the triggering inputs preconditions?}\\
 & {Fault, reboot, timeout, background protocols, and others}\\ 
%&{\scriptsize Node/Job crash}\\
%&{\scriptsize Node/Job restart after crash}\\
%&{\scriptsize Node/Job shut-down}\\
%&{\scriptsize Node/Job start-up}\\
%&{\scriptsize Time out}\\
%&{\scriptsize Others (e.g., client requests)}\\
\multicolumn{2}{l}{\it What is the triggering scope?}\\
&{\it How many nodes/messages/protocols are involved?}\\
\midrule
%\rowcolor{Gray}
\multicolumn{2}{c}{{\bf Errors \& Failures} }\\
\midrule
\multicolumn{2}{l}{\it What is the error symptom?}\\
&{Local memory exceptions}\\
&{Local semantic error messages \& exceptions}\\
&{Local hang}\\
&{Local silent errors (inconsistent local states) }\\
&{Global missing messages}\\
&{Global unexpected messages}\\
&{Global silent errors (inconsistent global states)}\\
\multicolumn{2}{l}{\it What is the failure symptom?}\\
&{Node downtimes, data loss/corruption, operation failures, slowdowns}\\
\midrule
%\rowcolor{Gray}
\multicolumn{2}{c}{{\bf Fixing} }\\
\midrule
\multicolumn{2}{l}{\it What is the fix strategy?}\\
&{Fix Timing: add global synchronization}\\
&{Fix Timing: add local synchronization}\\
&{Fix Handling: retry message handling at a later time}\\
&{Fix Handling: ignore a message}\\
&{Fix Handling: accepting a message without new computation logics}\\
&{Fix Handling: others}\\
\bottomrule
\end{tabular}
\mycaption[Taxonomy of DC Bugs]{tab:tax}{Taxonomy of DC Bugs}{}
\end{table}


We study the characteristics of DC
bugs along three key stages: triggering, errors \& failures, and
fixing (Table \ref{tab:tax}).
%
{\it Triggering} is the process where software execution states
deviate from correct to incorrect under specific conditions.  At the
end of this process, the manifestation of DC bugs changes from
non-deterministic to deterministic.
%
{\it Errors and failures} are internal and external software
misbehaviors.
%
{\it Fixing} shows how developers correct the bug.  We
will discuss in detail these categories in their respective sections.

% -----------------------------------------
\subsection{Threats to Validity}
\label{met-valid}

For every bug, we first ensure that the developers marked it as a real bug (not
a false positive).  We also check that the bug description is clear. Finally, We
then {\em re-enumerate} the full sequence of operations (the ``{\em steps}'') to
a clearer and more concise description such as the ones in Figure
\ref{fig-zook}. 
%
Our study cannot and does not cover DC bugs not fixed by the developers.  Even
for fixed bugs, we do not cover those that are not described clearly in the bug
repositories, a sacrifice we had to make to maintain the accuracy of our
results.  

Readers should be cautioned not to generalize the statistics we report as each
distributed system has unique purpose, design and implementation.
%
For example, we observe 2:1 overall ratio between order and atomicity violations
\sec\ref{trig-time}, however the individual ratios are different across the four
systems (\eg\ 1:2 in ZooKeeper and 6:1 in MapReduce).  
%
Like all empirical studies, our findings have to be interpreted with our
methodology in mind.

%\vfive % for good spacing

% -----------------------------------------
\subsection{TaxDC Database}
\label{met-db}

We name the product of our study \tdc\ database.  \tdc\ contains in
total \numTagsAll\ classification labels and \numDescLOC\ lines of
clear and concise re-description of the bugs (our version, that we
manually wrote) including the re-enumeration of the steps, triggering
conditions, errors and fixes.
%
We release \tdc\ to the public
\footnote{\url{http://ucare.cs.uchicago.edu/project/taxDC}}.  We believe \tdc\
will be a rich ``bug benchmark'' for researchers who want to tackle distributed
concurrency problems.  They will have sample bugs to begin with, advance their
work, and do not have to repeat our multi-people-year effort.


\vfive % for good spacing

% -----------------------------------------
\subsection{Detailed Terminologies}
\label{met-pres}

% ----------------- state
Below are the detailed terminologies we use in this chapter.
% basics
We use the term ``state'' to interchangeably imply {\em local state}
(both in-memory and on-disk per-node state) or {\em global state}
(a collection of local states and outstanding messages).
%
A {\em protocol} (\eg, read, write, load balancing) creates a chain of
events that modify system state.
%
User-facing protocols are referred as {\em foreground} protocols while
those generated by daemons or operators are referred
as {\em background} protocols.


% ------------------- fault
We consider four types of {\em events}: message, local computation,
fault and reboot.  The term {\em fault} represents component failures
such as crashes, timeouts, and disk errors.
%
A {\em timeout} (system-specific) implies a network disconnection
or busy peer node.
%
A {\em crash} usually implies the node experiences a power failure. %and does
%not come back up.
%
A {\em reboot} means the node comes back up.


% ---------------------------- DC bug presentation
Throughout the chapter, we present bug examples by abstracting
system-specific names.  As shown in Figure \ref{pat}, we use capital
letters for nodes (\eg, A, B), two small letters for a message between
two nodes (\mab\ is from A to B).  Occasionally, we attach
system-specific information in the subscript (\eg, A\sub{AppMaster}
sends \mab\sub{taskKill} message to B\sub{NodeManager}).
%
We use \underline{`` \textbf{/} ''} to imply concurrency
(\mac\ss\mbc\ implies the two messages can arrive at C in different
orders, \mac\ or \mbc\ first).
%
A dash, \underline{`` {\bf --} ''}, means causal relation of two
events (\mab-\mbc\ means \mab\ causally precedes\ \mbc).
%
Finally, we use \underline{``N{\bf *}''} to represent crash,
\underline{``N{\bf !}''} reboot, and \underline{``N{\bf +}''} local
computation at N.


% -------------------------- issue citation
We cite bug examples with clickable hyperlinks (\eg, \mr{3274}).
%
To keep most examples uniform, we use MapReduce examples whenever
possible.
%
%For interested readers, we cite more examples in the 
%footnotes (\eg, \spa\spb\spc).
%
%
We use the following abbreviations for system names:
``c/CA'' for Cassandra,
``h/HB'' for HBase, 
``m/MR'' for Hadoop MapReduce, and
``z/ZK'' for ZooKeeper;
% 
and for system-specific components:
``AM'' for application master,
``RM'' for resource manager,
``NM'' for node manager, 
``RS'' for region server, and
``ZAB'' for ZooKeeper atomic broadcast.


