
\begin{table}[!hbt]
\begin{center}
% {\small
\begin{tabular}{r|p{5in}}
\hline
{\bf Classification} & {\bf Labels} \\
\hline
Issue Type & Vital, miscellaneous. \\
\hline
Aspect & Reliability, performance, availability, security, consistency,  scalability, topology, QoS. \\
\hline
Bug scope & Single machine, multiple machines, entire cluster. \\
\hline
Hardware & Core/processor, disk, memory, network, node. \\
\hline
HW Failure & Corrupt, limp, stop. \\
\hline
Software & Logic, error handling, optimization, config, race, hang, space, load. \\
\hline
Implication & Failed operation, performance, component downtime, data loss, data staleness, data corruption. \\
\hline
\end{tabular}


%
\if 0
\begin{tabular}{p{3in}}
\\
\hline
{\bf Per-component Labels} \\
\hline
{\em Cassandra:} Anti-entropy, boot, client, commit log, compaction, cross system, get, gossiper, hinted handoff, IO, memtable, migration, mutate, partitioner, snitch, sstable, streaming, tombstone. \\
\hline
{\em Flume:} Channel, collector, config provider, cross system, master/supervisor, sink, source. \\
\hline
{\em HBase:} Boot, client, commit log, compaction, coprocessor, cross system, fsck, IPC, master, memstore flush, namespace, read, region splitting, log splitting, region server, snapshot, write. \\
\hline
{\em HDFS:} Boot, client, datanode, fsck, HA, journaling, namenode, gateway, read, replication, ipc, snapshot, write. \\
\hline
{\em MapReduce:} AM, client, commit, history server, ipc, job tracker, log, map, NM, reduce, RM, security, shuffle, scheduler, speculative execution, task tracker. \\
\hline
{\em Zookeeper:} Atomic broadcast, client, leader election, snapshot. \\
\hline
\end{tabular}
\fi
%---------------------------------
\end{center}
%\vminfifteen
%\vminten
\vminfive
\mycaption{tab-tag}{Issue Classifications}{
%
%  The tables list our issue classifications
%and per-component labels.
% the raw bug repositories have component labels, but they are
%unsuitable for our study (\eg, they are 
%too coarse-grained,  based on sub-directory names, and often
%do not pinpoint the buggy components).
}
%\vminfifteen
%\vminfive
\end{table}
