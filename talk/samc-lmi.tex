
\section{SAMC}

\subsection{Outline}

Hopefully, you get the intuition, and now let me show you the architecture of
SAMC

\subsection{SAMC Architecture}

Like other model checker.

SAMC intercepts all outstanding messages, suspend them, and decides the order in
which they will be released.

To reduce unnecessary re-orderings, SAMC is built on foundational techniques,
such dynamic partial order reduction (or DPOR) and symmetry.

On top of this, we build generic reduction policies such as local message
independence, .

But semantic awareness is about protocol-specific, so testers need to write
protocol-specific rules on top of the generic policies.

Now, let me describe the middle layer here.

\section{LMI}

\subsection{Local-Message Independence}

Lets begin with LMI.

It’s about detecting independence among local mes- sages in each individual
node, as I’ve explained ear- lier.

\subsection{Local-Message Independence}

As you have seen, discard pattern is one thing that help us.

We also find other generic patterns such as increment and constant update.
an example of increment pattern is a master that counts ack messages from
slaves.

You can see here, re-ordering ack messages don’t make us see a new state.

this pattern is common in quorum writes such as in zookeeper and cassandra.

so here we can write increment predicates that skip re-ordering of ack messages.

And there is also constant pattern that talk about in the paper.

