\section{CMI}

\subsection{Crash-Message Independence}

Okay, so far I haven't talked about crash yet, so let look at how recovery
semantic helps us to reduce crash re-orderings.

\subsection{Crash-Message Independence}

Look at distributed write protocol in this leader- follower architecture.

What happen if we crash the follower and there are outstanding messages ABCD to
the other followers?

well, a naive model checker must reorder the crash event and the messages.

But, if we use the recovery semantic knowledge, we know that the leader just
decrease follower count, but does not create new messages to the other nodes.

this is what we call as local impact

So we can write a rule like this that tells if the crash is on follower and we
have enough nodes, the crash is INDEPENDENT to the outstanding messages ABCD and
hence no need to reorder.

\subsection{Crash-Message Independence}

BUT, say we inject a crash on the leader, then this crash event will create new
messages, so this is a GLOBAL impact. so this crash MUST be reordered with other
messages, and this is a rule.

in summary, CMI uses recovery semantic to tell us ahead of time:

\begin{enumerate}
\item if crash leads to local impact crash is independent, so no re-ordering
\item if crash leads to global impact crash is dependent, must reorder
\end{enumerate}

\section{CRS and RSS}

The other policies are crash recovery symmetry and reboot synchronization
symmetry. Their detail is in the paper.

