
\section{Evaluation}

\subsection{Result}

We have built SAMC from scratch, and it took around 10,000 lines of code for all
mechanism.

We integrated to 7 protocols on 3 cloud systems, ZooKeeper, Cassandra,
MapReduce, including 10 version, new and old.

\subsection{Protocol-Specific Rules}

This is an example of protocol-specific rules from ZooKeeper leader election.

On average, it takes 35 lines of code for each protocols.

\subsection{Catching Old Bugs}

We can re-produce 12 old bugs that were reported by running SAMC with old
version of cloud systems.

We compare it with other state-of-the-art techniques, black-box DPOR that is
DPOR without domain-specific knowledge, random that is randomly re-ordering the
events, and random DPOR that is DPOR that start with random ordering.

Then we measure how many executions each model checker has to run until it reach
bugs.

By execution, I mean the ordering of events since the beginning up to
termination point or seeing some bugs.

This is SAMC.

This is black-box DPOR. This is random.

This is random DPOR.

5000+ means we have run 5000 executions but still not catch the bugs, and that
was around 2 days.

And here we compute speed-up.

Compare to black box, SAMC can do more than 94 times faster

And about 300 times faster compare to random and random DPOR.

\subsection{Reduction Ratio}

We also calculated reduction ratio compare to black- box DPOR on ZooKeeper
leader election protocol.

We also did fine-grained evaluation for each reduction policy too.

By trying different crashes and reboot, the result is shown in this table.

The reduction ratio increases when the number of crash and reboot increases. And
that means SAMC can find deep bugs faster than other techniques.

\subsection{Conclusion}

Today, we have cloud systems that quite complex, and deep bugs do happen to the
systems.

One approach to catch deep bugs is distributed system model checker, however,
the model checker treats the system as a black box and lead to state space
explosion

We are showing here that if we open the black box, make understanding of the
semantic, we can detect unnecessary ordering that lead to the same state, we
remove that and get one to two order of magnitude speedup

I believe that in the future we can use the principal of semantic awareness to
build more efficient reduction policies.

