\section{Integration}

Next, I'll show how to employ SAMC.

\subsection{Interposition layer}

To employ SAMC, we adopt aspect-oriented programming to insert interposition
layer. This makes the interposition layer code not clutter with software code.
SampleSys code contains no model checking code at all.

And here is the aspect for SampleSys, I intercept message at sending method.

To notify the server, we encapsulate message information as key-value pairs. The
neccessary key is message id, which I use the hash value of the message.

And to do white-box testing, I attach other information along with the
notification to the model checking server. 

Interposition layer communicates with the server via RPC.

%That is for interposition layer.

\subsection{SAMC server}

And for the server side, SAMC comes with basic ready-to-use algorithms like
brute-force or random.

But it also comes with extendable dynamic partial order reduction which is the
foundation of local-message independence, so we can extend it to work with
Sample by implementing isDependent method.

\subsection{Workload driver}
That's about writing exploration algorithm, the other things we need to do is
telling SAMC how to start and stop the system to do model check. And how to
determine the result after each execution.

User can do this by extending abstract classes that SAMC provides. 

For SampleSys, I make SAMC start three SampleSys processes when it starts
execution. And after each execution, SAMC checks whether every node agree on
one leader.

