
\section{Future}

\XXX{4 mins}

Well, that's all for my previous work. Now, I'll talk about my future work.

\subsection{Future Direction}

Here is my future direction

Toward DC bugs, TaxDC shows model checker needs to capture more events.

But to do that more challenges lie ahead, which I'll show.

For scalability bugs, they can happen in many axes, I just address one axe of
them. Many issues haven't been touched yet. I'll show you what are waiting for
us.

\subsection{DMCK}

Right now, SAMC doesn't address local computation timing, And also failure
model SAMC uses is only node fail-stop and reboot.

My first next step is advancing SAMC to capture local computation and expand
failure model.

The challenge here is this creates more events to re-order, and exacerbates
state space explostion.

So I will create more generic reduction policies to mitigate state space
explosion.

%In combating DC bugs, SAMC, a semantic aware model checker, leverage a domain
%specific knowledge to tackle state space explosion. 
%
%Right now, the generic patterns I exploit to build protocol-specific reduction
%policies are very simple. But this just scratches the surface of the power of
%semantic awareness. There are much more patterns and more reduction priciples
%waiting us to explore.
%
%So the next step on unearthing DC bugs is finding more patterns and creating
%more generic reduction policies, which will accelerate checking time.

%\subsection{AMC}

However, if you consider the current patterns I have here, they are very easy
to extract from the codebase.

Right now, we will have more reduction.

More reduction will find bugs faster, but that means extracting semantic
knowledge will be more complicated and harder.

This creates more burden on developers, and also introduce a chance of human
error.

And the error could leave bugs undetected..

My second step here will get rid of the burden of writing protocol-specific
policies by adopting program analysis to auto generate the policies.

%This will reduce the burden from developers and also the error.

\subsection{Scalability bugs}

Then let's look at scalability bugs. This class of bugs is new generation bugs
to combat in cloud-scale distributed systems.  And not much attention paid on
them. 

My vision here is to raise awareness on combating them.

On September this year, Cassandra developers started a new initiative to build
``Gossip 2.0'', by aiming at 1000 node cluster.

If they don't have a good way to test scalability, latent scalability bugs will
linger there. We might see a call for gossip 3.0 in a next few years.

%\subsection{4 axes}

As I said SCk is one of pilot works. It addresses only one axe of scalability,
which is scale of cluster.

From our cloud bug study, there are also

Scale of data

Scale of workload

And scale of failure

%SCk covers some part of scale of cluster. 

Previous work like Exalt covers some of scale of data. 

Many challenges lie ahead. In my future work, I'll adopt other techniques such
program analysis, to combat unsolved problems.

