\section{ZooKeeper Integration}

And that is about the demo program. Next, I will show you the result of
integration to Apache ZooKeeper.

\subsection{ZooKeeper}

In the real complex system, the non-determinism that could introduce bugs is not
only network re-ordering, but also disk I/O, or even machine crash. 

Most importantly, bugs always appear because the mix of these non-determinism.
Especially, they often occur in complex recovery scenarios.

In current version of SAMC, not only event re-ordering, it also injects
crashes and restarts nodes back to emulate failure and recovery scenario.

I model checked five versions of ZooKeeper, and can re-produce seven old bugs.
Some requires multiple crashes and multiple reboots to manifest.

Most importantly, SAMC can find a new bug on the latest version of ZooKeeper.

\subsection{ZooKeeper table}

This table shows the number of execution paths that each algorithm executed to
catch the bugs. I stopped at 5000 executions or around 2 days. Semantic-aware
uses domain-specific knowledge to prune out some executions so it can reach the
bugs very fast.

\section{Conclusion}

And that's all about SAMC. Here comes to the conclusion.

\subsection{SAMC}

The power of SAMC is semantic-aware that allows us to do fast  model checking.

We employ AOP to interpose interposition layer.

The server comes with basic algorithms and extendable advanced one.

We have successfully integrated SAMC to ZooKeeper to catch old and new bugs.

\subsection{Future}

And here is my future works.

SAMC now cannot deal with timeout-related bugs, so in the future, we plan to
make SAMC be able to inject timeout event.

Then we will make SAMC to be able catch performance bugs.

And the last thing is building step-by-step debugging replay function so
developers will work much more easier.

