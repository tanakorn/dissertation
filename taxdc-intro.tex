%\section{Introduction}
%\label{sec-intro}

Concurrency bugs are one notorious type of software bugs that happen in
concurrency systems. These timing-related bugs manifest non-deterministically,
and hence are extremely difficult to detect, diagnose, and fix. A huge body of
work exists in this space that focuses on ``local'' concurrency bugs (LC bugs)
in single-machine multi-threaded software, caused by incorrect interleaving of
memory accesses. And for cloud-scale distributed systems, the reliability is
also severely threatened by non-deterministic concurrency bugs as well, which
we refer as {\em distributed concurrency bugs} (DC bugs). Distributed systems
execute many complicated distributed protocols on hundreds/thousands of
machines with no common clocks, and must face a variety of random hardware
failures \cite{Do+13-Limplock, Gunawi+14-Cbs}. This combination makes
distributed systems prone to DC bugs caused by non-deterministic timing of
distributed events such as message arrivals, node crashes, reboots, and
timeouts. These DC bugs cannot be directly tackled by LC bug techniques, and
they cause fatal implications such as operation failures, downtimes, data loss
and inconsistencies.

Fighting DC bugs is challenging, particularly given the preliminary
understanding of real-world DC bugs.  To make progress, a comprehensive bug
study is needed. Past studies have closely examined bugs in various software
systems \cite{Chou+01-Empirical, Lu+13-FsEvolution, linux.asplos11}, which have
motivated and guided many aspects of reliability research.
%
There are few bug studies on cloud-scale distributed systems
\cite{Gunawi+14-Cbs, Li+13-ScopeBugStudy}, but they did not specifically dissect
DC bugs. There was an internal bug study dissecting network-failure-related DC
bugs to be a foundation to combat those bugs, but it was not published
\cite{Joshi+13-SetsudoTesting}, and one recent work analyzed non-determinism in
MapReduce programs but only discussed five bugs \cite{Xiao+14-NonDetMR}.
%
Thorough studies have also been conducted for LC bugs \cite{study.dsn10,
Lu+08-ConcurrencyBugStudy} with many follow-up work to date, yet {\em there is
no comprehensive study on real-world distributed concurrency bugs}. 

In this chapter, we fill the void by presenting our in-depth analysis of
real-world DC bugs in well-known cloud distributed systems, and introducing
\taxdc, the largest comprehensive taxonomy of DC bugs that covers several axes.
We briefly give an overview of \taxdc\ in Section \ref{sec-taxdc}, present our
analysis in Section \ref{sec-trig}-\ref{sec-stat}, and discuss some lessons we
learn from the study in Section \ref{sec-less}.

\if 0
We fill this void in this chapter by presenting our in-depth analysis of
\numDcBugs\ DC bugs.  The bugs came from four popular cloud distributed
systems: Cassandra \cite{CassandraWeb}, HBase \cite{HBaseWeb}, Hadoop MapReduce
\cite{HadoopWeb}, and ZooKeeper \cite{ZooKeeperWeb}.
%
We introduce \taxdc, a comprehensive taxonomy of real-world DC bugs across
several axes of analysis such as the triggering timing condition and input
preconditions, error and failure symptoms, and fix strategies, as shown in
detail in Table \ref{tab:tax}.
\fi
