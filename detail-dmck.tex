\section{Distributed System Model Checking}

One big challenge faced by a dmck is the state-space explosion problem (\ie,
there are too many distributed events to re-order). To address this, existing
dmcks adopt a random walk or basic reduction techniques such as dynamic partial
order reduction (DPOR). Despite these early successes, existing approaches
cannot unearth many real-world DC bugs, so we are advancing the state of the art
of dmck to combat DC bugs, which described next.

\subsection{Semantic-Aware Model Checking (Initial Work)}

We started my work by specifically addressing two limitations of existing dmcks.
First, existing dmcks treat every target system as a complete \textit{black
box}, and therefore perform unnecessary reorderings of distributed events that
would lead to the same explored states (\ie, redundant executions). Second, they
do not incorporate complex multiple fault events (\eg, crashes, reboots) into
their exploration strategies, as such inclusion would exacerbate the state-space
explosion problem.

To address these limitations, we introduced Semantic-Aware Model Checking (SAMC)
\cite{Leesatapornwongsa+15-SamcIssta, Leesatapornwongsa+14-Samc}, a novel
white-box model checking approach that takes \textit{semantic knowledge} of how
distributed events (specifically, messages, crashes, and reboots) are processed
by the target system and incorporates that information in reduction policies.
The policies are based on sound reduction techniques such as DPOR and symmetry.
The policies tell SAMC not to re-order some pairs of events such as
message-message pairs or crash-message pairs, yet preserves soundness, because
those cut out re-orderings are redundant.

\subsubsection{Semantic-Aware Reduction Policies}

\begin{itemize}

\item {\bf Local-Message Independence (LMI):} A black-box DPOR treats the
message processing inside the node (local message) as a black box, and thus must
declare the incoming messages to the same node as dependent. We propose LMI that
if we have semantic knowledge about message processing, we can define
independency relationship among local messages. LMI prune out state space by not
reordering independent local messages.

\item {\bf Crash-Message Independence (CMI):} When crash happens, black-box DPOR
expects that every node will react to the crash (recovery) and consider that the
crash is independent to every current on-going messages. But CMI suggests that
some crashes could be considered independent and avoid reordering those crashes.

\item {\bf Crash Recovery Symmetry (CRS):} While injecting crash, dmck needs to
make sure that it covers every crash scenario by trying every possible crash
(\ie\ try crashing every possible node). However, at a particular time, the
systems could recover from two different crashes in the same manner. CRS claims
that dmck can explore only one crash recovery scenario, and prune out the
others.

\item {\bf Reboot Synchronization Symmetry (RSS):} Same as CRS, dmck does not
need to try reboot for every possibility, if the reboot does not lead to any
different behavior.

\end{itemize}

\subsubsection{Result}



\def \fth {\uuu 5000}



\begin{table*}[!hbt]
\begin{center}
{\small
%---------------------------------
\begin{tabular}{l|l|ccc|rrrr|rrr} 

 {} & {} &
 {} & {} & {} &
\multicolumn{4}{c|}{{\bf \#Executions }} &
\multicolumn{3}{c}{{\bf Speed-up of SAMC vs. }} \\


{\bf Old Issue\#}  &  {\bf Protocol} & 
{\bf E} & {\bf C} & {\bf R} &
\bf{bDP} & \bf{RND} & \bf{rDP} & \bf{SAMC} &
\bf{bDP} & \bf{RND} & \bf{rDP} \\

\hline
\zk{335}   &   ZAB  & 120 & 3 & 3  &  \fth &  1057 &  \fth &   117 &  \uu 43 &      9 & \uu 43 \\
\zk{790}   &   ZLE  &  21 & 1 & 1  &    14 &   225 &    82 &     7 &       2 &     32 &     12 \\
\zk{975}   &   ZLE  &  21 & 1 & 1  &   967 &    71 &   163 &    53 &      18 &      1 &      3 \\
\zk{1075}  &   ZLE  &  25 & 3 & 2  &  1081 &    86 &   250 &    16 &      68 &      5 &     16 \\
\zk{1419}  &   ZLE  &  25 & 3 & 2  &   924 &  2514 &   987 &   100 &       9 &     25 &     10 \\
\zk{1492}  &   ZLE  &  31 & 1 & 0  &  \fth &  \fth &  \fth &   576 &   \uu 9 &  \uu 9 &  \uu 9 \\
\zk{1653}  &   ZAB  &  60 & 1 & 1  &   945 &  3756 &  3462 &    11 &     86  &    341 &    315 \\
\hline
\mr{4748}  &    SE  &  25 & 1 & 0  &    22 &     6 &     6 &     4 &       6 &      2 &      2 \\
\mr{5489}  &    CM  &  20 & 2 & 1  &  \fth &  \fth &  \fth &    53 &  \uu 94 & \uu 94 & \uu 94 \\
\mr{5505}  &    CM  &  40 & 1 & 1  &  1212 &  \fth &  1210 &    40 &      30 & \uu125 &     30 \\
\hline
\cs{3395}  & RW+HH  &  25 & 1 & 1  &  2552 &   191 &   550 &   104 &      25 &      2 &      5 \\
\cs{3626}  &    GS  &  15 & 2 & 1  &  \fth &  \fth &  \fth &    96 &  \uu 52 & \uu 52 & \uu 52 \\
\end{tabular}
}
%\vminten
%---------------------------------
\end{center}
\mycaption{tab-oldbugs}{SAMC Speed in Finding Old Bugs}{
%
``E'', ``C''
and ``R'' represent the number of events, crashes, and reboots
necessary to hit the bug.  The numbers in the middle four columns
represent the number of executions to hit the bug across different
policies.  ``bDP'', ``RND'', and ``rDP'' stand for black-box DPOR (in
\modist), random, and random + black-box DPOR respectively.
%Numbers marked with ``\nu'' imply that the experiments are still running
%at the time of this paper submission and the bug has not been reached.
We stop at 5000 executions (around 2 days) if the bug
cannot be found (labeled with ``\uuu'').  
Thus, speed-up numbers marked with
``\uu'' are potentially much higher.  
%
}

%\vminfive

\end{table*}

\if 0
\zk{1653}  &    --  & --- & 1 & 1  &    -- &    -- &    -- &    -- &     --- & --- & --- \\
\fi


SAMC is able to reproduce 12 old bugs in three cloud systems (Cassandra, Hadoop
MapReduce, and ZooKeeper) involving 30-120 distributed events and multiple
crashes and reboots. Some of these bugs cannot be unearthed by non-SAMC
approaches, even after two days. SAMC can find the bugs up to 271x (33x on
average) faster compared to state-of-the-art techniques as shown in Table
\ref{tab-oldbugs}. Additionally, we found two new bugs in Hadoop MapReduce and
ZooKeeper.

