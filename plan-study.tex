\section{Bug Study}

Bug or failure studies can significantly guide many aspects of dependability
research. Many dependability researchers have recently employed formal studies
on bugs and failures such as the studies on large-scale system bugs/failures
from Microsoft \cite{Guo+13-CureIsWorse, Li+13-ScopeBugStudy} These studies can
identify opportunities for new research, build taxonomies of new problems, or
test new tools. I started my research by doing formal bug study to gain
foundations of combating DC bugs.

\subsection{Cloud Bug Study}

As an initiative, our group have performed the largest bugs study in six 
important Apache cloud infrastructures including Cassandra, Flume, Hadoop
MapReduce, HBase, HDFS, and ZooKeeper \cite{Gunawi+14-Cbs}. We reviewed in
total 21,399 submitted issues within a three-year period (2011-2014) in Apache
bug repositories. We perform a deep analysis of 3,655 ``vital'' issues (\ie,
real issues affecting deployments) with a set of detailed classifications. This
work led us to several interesting dependability research questions, and was 
the main source of my DC-bug taxonomy work.

\subsection{DC Bug Taxonomy} 

While there have been many LC-bug studies, I am not aware of any large-scale
study of DC bugs. A recent study from Microsoft analyzed the effect of
distributed concurrency on workload and only studied five DC bugs in MapReduce
systems \cite{Xiao+14-NonDetMR}. To fill the void, I as one of the project
leaders, have created the largest and most comprehensive taxonomy of 104
real-world DC bugs (named \taxdc) from Cassandra, HBase, Hadoop MapReduce/Yarn,
and ZooKeeper \cite{Leesatapornwongsa+16-TaxDC}. \taxdc\ contains in-depth
characteristics of DC bugs, stored in the form of 2,083 classification labels
and 4,528 lines of re-enumerated steps to the bugs that I manually added.
Motivated by the availability of bug benchmarks for LC bugs, I will release
\taxdc\ as a large-scale DC bugs benchmark.

With \taxdc\, I can answer important questions such as: How often are DC bugs
reported from real deployments? What types of DC bugs exist in real world?
What are the root causes of DC bugs (out-of-order messages, failures, \etc)?
Are existing LC-bug-detection tools applicable for DC bugs? How do developers
fix DC bugs (by adding locks, states, \etc)? What are the inputs/triggering
conditions?  What are the minimum number of distributed events needed to
trigger the bugs (how many messages to re-order, failures to inject, \etc)?
What errors/effects (specification violations) are caused by DC bugs (deadlock,
data loss, state inconsistency, performance problems, \etc)? How do propagation
chains form from the root causes to errors? The answers to these questions will
guide my subsequent research projects.

