\section{Distributed Concurrency Bugs}

Distributed concurrency bugs (DC bugs) are bugs that caused by nondeterministic
orders of distributed events. Distributed events could be message arrivals,
hardware crashes/reboots, network timeout, \etc\ Cloud systems execute multiple
complicated distributed protocols concurrently (\eg, serving users' requests,
operating background tasks, and combined with untimely hardware failures), and
possible interleavings of the distributed events are beyond developers'
anticipation, which some interleavings might not be handled properly, and can
cause catastrophic failures such as data loss/inconsistency and downtimes.
Compared to the ``countless'' of efforts in combating ``local'' concurrency bugs
in multi-threaded software, DC bugs have not received the same amount of
attention within the research community.

Here is our contributions to combat DC bugs in systematic and comprehensive manners,

\begin{enumerate}

\item Semantic-Aware Model Checking (SAMC): we advance the state of the art of
model checking for distributed systems by adopting white-box approach to tackle
state-space explosion, the current limitation of model checking.

\item Taxonomy for DC bugs (\taxdc): we perform an in-depth study of more than
100 real-world DC bugs and built a first complet taxonomy for DC bugs. This
study will give insight to guide many future research work on DC bugs.

\end{enumerate}

The brief detail of these two works are discussed below.

\subsection{Semantic-Aware Model Checking}

One powerful method for discovering hidden DC bugs is the use of an
\textit{implementation-level distributed system model checker} (\textbf{dmck}).
A dmck can discover buggy interleavings that lead to DC bugs by reordering every
possibility of nondeterministic distributed events. The last ten years have seen
a rise of dmcks such as MaceMC, \modist, or Demeter. One big challenge faced by a
dmck is the state-space explosion problem (\ie, there are too many distributed
events to re-order). To address this, existing dmcks adopt a random walk or
basic reduction techniques such as dynamic partial order reduction (DPOR).
Despite these early successes, existing approaches cannot unearth many
real-world DC bugs, so we advance state of the art of dmck to combat DC bugs.

We start by addressing two limitations of existing dmcks. First, existing dmcks
treat every target system as a complete \textit{black box}, and perform
unnecessary reorderings of distributed events that would lead to the same states
(\ie, redundant executions). Second, they do not incorporate complex multiple
fault events (\eg, crashes, reboots, \etc) into their exploration strategies, as
such inclusion would exacerbate the state-space explosion problem.

To address these limitations, we introduce Semantic-Aware Model Checking
(\textbf{SAMC}) \cite{Leesatapornwongsa+15-SamcIssta,Leesatapornwongsa+14-Samc},
a novel white-box model checking approach that takes \textit{semantic knowledge}
of how distributed events (specifically, messages, crashes, and reboots) are
processed by the target system and incorporates that to create reduction
policies. The policies are based on sound reduction techniques such as DPOR and
symmetry. The policies tell SAMC not to re-order some pairs of events such as
message-message pairs, and message-crash pairs, yet preserves soundness, because
those cut out re-orderings are redundant, and unnecessary to check.

SAMC can reproduce twelve old bugs in three cloud distributed systems
(Cassandra, Hadoop MapReduce, and ZooKeeper) involving 30-120 distributed events
and multiple crashes and reboots. Some of these bugs cannot be unearthed by
non-SAMC approaches, even after two days. SAMC can find the bugs up to 340 (49x
on average) faster compared to state-of-the-art techniques, it found two new
bugs in Hadoop MapReduce and ZooKeeper.

\subsection{DC Bug Study \& Taxonomy}

Bug and failure studies can significantly guide many aspects of dependability
research. Many researchers have recently employed formal studies on bugs and
failures \xxx{gather citation from research statement}.
%
However, we are not aware of any public large-scale DC-bug study, a recent study
from Microsoft analyzed the effect of distributed concurrency of workload and
only studied five DC bugs in MapReduce \xxx{include NEC work}.

To fill the void, we create the first largest taxonomy of 104 real-world DC bugs
(named \taxdc) \cite{Leesatapornwongsa+16-TaxDC} from four popular large-scale
distributed systems: Cassandra, HBase, Hadoop MapReduce/Yarn, and ZooKeeper.
\taxdc\ contains in-depth characteristics of DC bugs, stored in the form of
2,083 classification labels and 4,528 lines of re-enumerated steps to the bugs
that we manually added. We show that although DC bugs are complex, but they can
be abstracted and categorized with our taxonomy that covers all aspects of DC
bugs such as bug triggering, errors and failures, and bug fixing.

\xxx{i might add more about taxdc}

