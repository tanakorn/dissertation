
% samc
In this thesis, we present semantic-aware model checking (SAMC;
pronounced ``Sam-C''), a white-box principle that takes simple semantic
information of the target system and incorporates that knowledge in
state-space reduction policies.    In our
observation, existing dmcks treat every target system as a complete
black box, and therefore many times perform message re-orderings and
crash/reboot injections that lead to the same conditions that have
been explored in the past.  These {\em redundant executions} must be
removed significantly to tame the state-space explosion problem.  
We find that simple semantic knowledge can
scale dmck greatly.

% main challenge
The main challenge of SAMC is in defining {\em what} semantic
knowledge can be valuable for reduction policies and {\em how} to
extract that information from the target system.  We find that useful
semantic knowledge can come from {\em event processing semantic}
(\ie, how messages, crashes, and reboots are processed by the
target system).  To help testers extract such information from
the target system, we provide {\em generic event processing patterns},
patterns of how messages, crashes, and reboots are processed by
distributed systems in general.


% policies and users
With this method, we introduce four novel semantic-aware reduction
policies.  First, {\em local-message independence} (LMI) reduces
re-orderings of concurrent intra-node messages.  Second, {\em
crash-message independence} (CMI) reduces re-orderings of crashes
among outstanding messages.  Third, {\em crash recovery symmetry}
(CRS) skips crashes that lead to symmetrical recovery
behaviors.  Finally, {\em reboot synchronization
symmetry} (RSS) skips reboots that lead to symmetrical
synchronization actions.  Our reduction policies are {\em generic};
they are applicable to many distributed systems.  SAMC users
(\ie, testers) only need to feed
the policies with short {\em protocol-specific rules} that
describe event independence and symmetry specific to their target
systems.


% systematic
SAMC is purely systematic; it does not incorporate randomness or
bug-specific knowledge.  Our policies run on top of sound model
checking foundations such as state or architectural
symmetry~\cite{Clarke+98-SymReduct, Prasad+00-SymBasedMc} and
independence-based dynamic partial order reduction
(DPOR)~\cite{Flanagan+05-Dpor, Godefroid+96-Dpor}.  Although these
foundations have been around for a decade or more, its application to
dmck is still limited; these foundations require testers to define
{\em what} events are actually independent or symmetrical.  With SAMC,
we can define fine-grained independence and symmetry.


% c3) integration, and also policies, protocols !!!!!!
We have built a prototype of SAMC (\sampro) from scratch for a total
of \numLinesSamPro\ lines of code.  We have integrated \sampro\ to
three widely popular cloud systems,
ZooKeeper~\cite{Hunt+10-ZooKeeperPaper},
Hadoop/Yarn~\cite{Kumar+13-Yarn}, and
Cassandra~\cite{Lakshman+09-Cassandra} (old and latest stable
versions; \numVersions\ versions in total).  We have run \sampro\
on \numProtocols\ different protocols (leader election, atomic
broadcast, cluster management, speculative execution, read/write,
hinted handoff, and gossiper).  The protocol-specific rules are
written in only \numLinesRule\ LOC/protocol on average.  This shows
the simplicity of applying SAMC reduction policies across different
systems and protocols; all the rigorous state exploration and
reduction are automatically done by \sampro.

% evaluation, 3 crashes, 3 reboots ... 
To show the power of SAMC, we perform an extensive evaluation of
SAMC's speed in finding deep bugs.  We take \numOldDeepBugs\ old
real-world deep bugs that require multiple crashes and reboots (some
involve as high as 3 crashes and 3 reboots) and show that SAMC can
find the bugs 
% up \numAvgBugSpeedUp\ faster on
% to \numMaxBugSpeedUp\ faster (
one to two orders of magnitude faster compared to state-of-the-art
techniques such as black-box DPOR, random+DPOR, and pure random.  We
show that this speed saves tens of hours of testing time.  More
importantly, some deep bugs cannot be reached by non-SAMC approaches,
even after 2 days; here, SAMC's speed-up factor is potentially much
higher.  We also found \numNewBugs\ new bugs in the latest version of
ZooKeeper and Hadoop.


% summarize
To the best of our knowledge, our work is the first solution that
systematically scales dmck with the inclusion of failures.  We believe
none of our policies have been introduced before.  Our prototype is
also the first available dmck for our target systems.  Overall, we
show that SAMC can address deep reliability challenges of cloud
systems by helping them discover deep bugs faster.

% the rest
The rest of the thesis is organized as follows.  First, we present a
background and an extended motivation (\sec\ref{sec-mot}).  Next, we
present SAMC and our four reduction policies (\sec\ref{sec-samc}).
Then, we describe \sampro\ and its integration to cloud systems
(\sec\ref{sec-impl}).  Finally, we close with evaluations
(\sec\ref{sec-eval}), related work (\sec\ref{sec-related}),
conclusion (\sec\ref{sec-conclude}), and future 
work(\sec\ref{sec-future}).

