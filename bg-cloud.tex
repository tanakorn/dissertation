\section{Cloud Computing and Cloud-Scale Distributed Systems}
\label{bg-cloud}

\subsection{Cloud Computing}

In the past decade, \textit{cloud computing} has become widespread buzzword
that IT people often talk about. However, there is few agreement on what it
really means; many companies and experts have given many definitions
\cite{TwentyoneCloudDef, IBMCloudDef, PCMagCloudDef, Foster+08-CloudAndGrid}
that are different in details, but they share some common characteristics: cloud
computing is on-demand internet-based services that can scale to serve growing
users' requests, and users \textit{pay as they go}.

% probably write about 3 models of cloud SaaS, PaaS, IaaS

%benefit of cloud computing
Cloud computing can attract a number of users to move their data and
computation from local machines and private datacenters to the cloud
\cite{AdobeCloudStat, AWSCustomer, GmailStat, GoogleDriveStat, DropboxStat,
AstroInCloud, FacebookStat, Luo+16-BigDataBioResearch}.
It provides many benefits as shown below:
\begin{itemize}
\item Users can access their computing resources and data any time and any
where. For example, Google Doc allows users to create/modify their documents on
one machine and access them later on mobile devices. This advantage also
enables new classes of applications, especially, mobile applications
\cite{DropboxWebsite, GmailWebsite, GoogleDriveWebsite, iCloudWebsite,
SiriWebsite} 

\item Users can cut the cost of hardware investment, but can get immediate access
to computing resources. Moreover, this will help improve hardware utilization
because users do not need to pay while they are not using the resources
\cite{Marston+11-CloudBusiness}.

\item Most importantly, cloud computing help users to scale their services in a
convenient manner. When their workload is growing up, they can just purchase
more computing power from their cloud service providers.
\end{itemize}

\subsection{Cloud-Scale Distributed Systems}

Behind the cloud computing, it is sophisticated distributed systems running on
large-scale clusters. Unlike traditional computing, when the number of users or
workload increase, we do not upgrade hardware specification, but add more
machines to the systems. Thus, cloud-scale distributed systems are designed to
be scalable to handle tremendous users' requests. Moreover, the cluster is
built from commodity machines which hardware failures are not optional and can
be very complex (\eg multiple machine failures, disk failures, and network
partitions), so cloud systems must be able to handle these complex failures
\cite{Abadi09-Cloud, Gunawi+11-FaaS-TR, Hamilton07-Deploying}.  We will discuss
some types of cloud-scale distributed systems below as these systems are
systems that we work on in this dissertation.

\begin{itemize}

\item \textit{Distributed file systems}: these are file systems that stores
files across machines in clusters (however, some systems cannot be mounted as
other traditional file systems, and some users consider them as data store
\cite{HadoopStorage}). Distributed file systems do replication or erasure coding
across machines in order to be fault tolerant, and to increase performance via
data aggregation. Examples of these systems are GFS from Google
\cite{Ghemawat+03-GoogleFS}, and HDFS from Apache \cite{Shvachko+10-HDFSPaper}
which is an open-source version of GFS.

\item \textit{Data-parallel framework}: these frameworks process \textit{big
data} by leveraging parallelism. It eases parallel computing by enabling users
to increase computing power by just adding more machines without changing their
programs.  One of well-known data-parallel frameworks is MapReduce from Google
\cite{DeanGhemawat04-MapReduce}.  As its name states, MapReduce is a
programming model that consist of \textit{map} and \textit{reduce} functions.
Map functions process key-value pairs of data and generate intermediate
key-value pairs, which reduce functions will process values of data with the
same keys to generate final results. Apache also has open-source framework,
which is similar to Google MapReduce, called Hadoop MapReduce \cite{HadoopWeb}.

\item \textit{NoSQL data stores}: These are data stores that are not relational
database. Storing and accessing data do not have strict tabular relations like
in relational database, and are not done by SQL query (some systems support SQL
query but not fully), such as key-value stores, document stores, and object
stores. Most NoSQL stores adopt the concept of ``eventual consistency'' to
improve availability during network partition (favoring ``C'' in CAP theorem). An
example of NoSQL is Dynamo from Amazon \cite{DeCandia+07-Dynamo}. Dynamo is
decentralized distributed key-value store. Its open-source counterpart from
Apache is Cassandra \cite{Lakshman+09-Cassandra}.

\item \textit{Synchronization services}: These are utility services that support
other large distributed systems. They help nodes in other systems synchronize
some metadata such as global locking, configuration maintaining, and naming.
Examples of the synchronization services are Chubby from Google
\cite{Burrows06-Chubby} and ZooKeeper from Apache \cite{Hunt+10-ZooKeeperPaper}.

\end{itemize}
