

\section{Introduction}

\hsg{Cesar, focus on program analysis.}

\hsg{Korn, focus on the bugs}

% Burrows06-Chubby, 
% Murray+13-NaiadTimelyDataflow,

\hsg{focus on the program analysis, making this automated.
remove the things that are not automated, and run on more machines.}

\hsg{add Tian Chen if funded}

\hsg{add Shan Lu}

% scale, scale ... 
Is scale a friend or a foe \cite{Ousterhout+11-ScaleFriendEnemy}?
% CACM, is scale friend or enemy, john ousterhout
On the positive side, scale surpasses the limit of a single machine in
meeting users' increasing demands of compute and storage, which led to
many inventions of ``cloud-scale'' distributed systems
\cite{Chang+06-BigTable, 
DeanGhemawat04-MapReduce, 
DeCandia+07-Dynamo,
Ghemawat+03-GoogleFS, 
Hindman+11-Mesos,
Verma+15-Borg}.  The field has witnessed a
phenomenal deployment scale of such systems;
%
Netflix runs tens of 500-node Cassandra clusters \cite{RunningNetflix13},
% Running Netflix on Cassandra in the Cloud (youtube), Adriak Crockcroft
% https://www.youtube.com/watch?v=97VBdgIgcCU
Apple deploys a total of 100,000 Cassandra nodes \cite{WikiCassandra}, 
% https://en.wikipedia.org/wiki/Apache_Cassandra
and Yahoo! recently revealed the use of 40,000 Hadoop servers,
with a 4500-node cluster as the largest one \cite{LargestHadoop}.
% Http://www.techrepublic.com/article/why-the-worlds-largest-hadoop-installation-may-soon-become-the-norm/



% dark side, foe
On the negative side, scale creates new development and deployment issues.
Developers must ensure that their algorithms and protocol designs
to be scalable.
However, until real deployment takes place, unexpected bugs 
in the actual implementations are unforeseen.
% more and more
We believe this new era of cloud-scale distributed systems has given birth
to a new type of bug: {\em scalability bugs}.  They are latent bugs that
are scale-dependent; they only surface in large-scale deployments, but not
in small/medium-scale ones.  Their presence jeopardizes systems
reliability and availability at scale.


% example
We study several scalability bugs that have been reported in distributed
P2P (no-master) key-value stores such as Cassandra
\cite{Lakshman+09-Cassandra}, Couchbase \cite{CouchbaseWeb}, Riak
\cite{RiakWeb}, and Voldemort \cite{VoldemortWeb}.  We focus ourselves on
scalability bugs in the ``control planes'' such as the bootstrap,
rebalance, and scale-out/down protocols (\sec\ref{mot-control}).  We make
several important observations from this study (\sec\ref{mot-observe}).
%
For example, first, the bug symptoms surface only in large deployment
scales (\eg, $N$$>$100 nodes).
%
Second, the protocol algorithms are scalable in design sketch, but not in
practice; there are specific implementation choices whose impacts at scale
are unpredictable.
%
Third, the developers had difficulties in reproducing and debugging the
bugs; they took weeks or months to debug and fix.



Our observations above accentuate the need for new approaches that can
{\em scale-check distributed system implementations at real scales but in
  a ``cheap'' way such as on one machine}.  Existing scale-check practices
however do not meet this need (\sec\ref{mot-state}).  For example,
scalability simulation only checks models, but not real implementations.
Extrapolation (\eg, from ``mini clusters'') does not work if bug symptoms
do not surface in small deployment.
%
Debugging at the same scale as customers' deployments (\eg, 500 nodes) is
not economical; in our observation, open-source developers do not have
instant and easy access to large test clusters.


% ; renting 500 ``medium'' EC2 machines can easily
% burn \$25 in one hour (not to mention the I/O cost).
%
% There are not many available solutions out there to address the need.
% Some recent work (\eg, Exalt \cite{exalt}) 
