\section{Scalability Bugs}

Scalability bugs are a type of bug that newly born in the era of cloud
computing. These bugs are latent such that they do not surface in
small/medium-scale deployments, but only surface in large scale. They threaten
systems reliability and availability at scale. As we discussed above, cloud
backend needs to be scalable; algorithms and protocols in cloud distributed
systems are designed to be scalable. However, until real deployment takes place,
if developers do not have a large cluster to test their actual implementations,
unexpected bugs are unforeseen. 

We see that most of the work \cite{Calotoiu+13-ApmScaleBug,
Laguna+15-DebugAtScale, Shudler+15-ExascaleLib, Wang+14-Exalt, Zhou+11-Vrisha,
Zhou+13-Wukong} focuses on the data path, mainly to validate the scalability of
read/write operations (linear throughput or stable latency as the cluster
scales). But scalability correctness however is not merely about the data path.
Distributed systems are full of ``control paths'' such as bootstrapping,
rebalancing, and adding/decommissioning nodes (scaling out/down). These
management protocols must modify cluster-wide metadata that lives in each node
in the system (\eg, ring partition table) to decide how data flows in the
cluster. Unfortunately, control path correctness is often overlooked, so in this
dissertation, we aim our attention to ``{\em control-plane scalability bugs}''.


