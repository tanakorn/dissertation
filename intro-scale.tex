\section{Scalability Bugs}

Scalability bugs are a type of bug that newly born in the era of cloud
computing. These bugs are latent such that they do not surface in
small/medium-scale deployments, but only surface in large scale. They threaten
systems reliability and availability at scale. As we discussed above, cloud
backend needs to be scalable; algorithms and protocols in cloud distributed
systems are designed to be scalable. However, until real deployment takes place,
if developers do not have a large cluster to test their actual implementations,
unexpected bugs are unforeseen. 

The following is our contribution to tackle this novel type of bugs:

\begin{enumerate}

\item Scalability bug study (SCB): we perform an in-depth study of 41
scalability bugs to analyze how an era of cloud computing gives a birth to a new
type of bugs that is scale dependent. This study is a bug benchmark for future
research on scalability aspect of cloud-scale distributed systems.

\item Scalability checking methodology for cloud-scale distributed systems
(\sck): we propose a methodology to help developers test and debug scalability
of systems in an economical way by colocate multiple nodes on one machine.

\end{enumerate}

The next sections discuss some detail of these two works.

\subsection{Scalablity Bug Study}

For scalability bugs, the situation is worse. We are not aware of any study on
scalability bugs at all, so in this dissertation, we start a study of
scalability bugs to gain some foundational knowledge about them. We studied 41
bugs in seven systems including Cassandra, Couchbase, Hadoop MapReduce, HBase,
HDFS, Riak, and Voldemort.

From the study, here is our observations regard to scalability bugs:

\begin{itemize}
\item Scalability bugs only appear at extreme scale.
\item Systems can be scalable in design, but not in practice.
\item Scalability bugs could be implementation specific and hard to predict.
\item Scalability bugs are caused from cascading impacts of ``not independent'' nodes.
\item It is long and difficult to debug large-scale.
\item Not all developers have large test budgets.
\end{itemize}

%\subsection{Scalability Checking Methodology}

%\xxx{Fill this after write sck chapter}

\if 0
We see that most of the work \cite{Calotoiu+13-ApmScaleBug,
Laguna+15-DebugAtScale, Shudler+15-ExascaleLib, Wang+14-Exalt, Zhou+11-Vrisha,
Zhou+13-Wukong} focuses on the data path, mainly to validate the scalability of
read/write operations (linear throughput or stable latency as the cluster
scales). But scalability correctness however is not merely about the data path.
Distributed systems are full of ``control paths'' such as bootstrapping,
rebalancing, and adding/decommissioning nodes (scaling out/down). These
management protocols must modify cluster-wide metadata that lives in each node
in the system (\eg, ring partition table) to decide how data flows in the
cluster. Unfortunately, control path correctness is often overlooked, so in this
dissertation, we aim our attention to ``{\em control-plane scalability bugs}''.
\fi
