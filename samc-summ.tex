

\subsection{Does State of the-Art Help?}
\label{mot-summ}

We now combine our observations in the two previous sections and
describe why state-of-the-art dmcks do not address present reliability
challenges of cloud systems.



% policies and bugs, random
First, {\em existing systematic reduction policies often cannot find
  bugs quickly}.  Experiences from previous dmck developments suggest
that significant savings from sound reduction policies do not always
imply high bug-finding effectiveness~\cite{Guo+11-Demeter,
  Yang+09-Modist}.  To cover deep states and find bugs, many dmcks
revert to non-systematic methods such as randomness or manual
checkpoints.  For example, \modist\ combines DPOR with random walk to
``jump'' faster to a different area of the state space (\sec4.5
of~\cite{Yang+09-Modist}).  DIR developers find new bugs by manually
setting ``interesting'' checkpoints so that future state explorations
happen from the checkpoints (\sec5.3 of~\cite{Guo+11-Demeter}).  In
our work, although we use different target systems, we are able to
reproduce the same experiences above (\sec\ref{eval-oldbugs}).



% multiple failures
Second, {\em existing dmcks do not scale with the inclusion of failure
  events}.  Given the first problem above, exercising multiple
failures will just exacerbate the state-space explosion problem.  Some
frameworks that can explore multiple failures such as
\macemc~\cite{Killian+07-LifeDeathMaceMC} only do so in a random way;
however, in our experience (\sec\ref{eval-oldbugs}), randomness many
times cannot find deep bugs quickly.  \modist\ also enabled only one
failure.  In reality, multiple failures is a big reliability threat,
and thus must be exercised.


We conclude that finding systematic (no random/checkpoint) policies
that can find deep bugs is still an open dmck research problem.  We
believe without semantic knowledge of the target system, dmck hits a
scalability wall (as also hinted by DIR authors; \sec8
of~\cite{Guo+11-Demeter}).  In addition, as crashes and reboots need
to be exercised, we believe recovery semantics must be incorporated into
reduction policies.  All of these observations led us to SAMC, which
we describe next.


