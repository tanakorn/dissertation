\section{Conclusion}

In this chapter, we have presented our observations on scalability bugs that
highlight a need of an attention to combat them. Scalability bugs are latent
bugs that are scale-dependent that only manifest in large scale. We have also
presented our pilot work \sck, a methodology to enable developers to colocate
hundreds of nodes on one machine to emulate large-scale deployments; this helps
developers save cost of testing and speed up testing process. We have introduced
four techniques in this chapter:

\begin{enumerate}

\item {\bf Processing Illusion} (PIL) helps reduce CPU contention so
CPU-intensive nodes can be more colocated on one machine.

\item {\bf Single Process Cluster} (SPC) helps reduce memory consumption and
context switching.

\item {\bf Global Event Driven Architecture} (GEDA) helps reduce the number of
threads we need to run the systems.

\item {\bf Memory Footprint Reduction} (MFR) helps reduce memory consumption
further.

\end{enumerate}

\sck is a pilot work, it still needs a lot of manual efforts from developers to
do scale check, but we hope that this work can raise awareness from system
community to pay attention to this new class of bugs in cloud-scale distributed
systems.
