

\section{Implementation and Integration}
\label{sec-impl}



%To make them scale-checkable on one machine, we added \locCass, \locRiak,
%and \locVold LOC respectively.




\begin{table}[t]
\begin{center}
\small
\centering
%---------------------------------
\begin{tabular}{l|r|r|r} 
{\bf } & {\bf Cassandra} & {\bf Riak} & {\bf Voldemort} \\
\hline
System               &    95757     &   18250      &  63485        \\
Unit Tests           &    18891     &   1341       &  15820        \\
\hline
(a) $+$ \sck-able        & \locCassMod  & \locRiakMod  &  \locVoldMod  \\
(b) $+$ \ts{sck} (tool)  & \locCassTool & \locRiakTool &  \locVoldTool \\
\hline
Total changes (\%)        &      5\%     &   4\%        &   2\%  
% $+$ \sck      &   \locCass  &   \locRiak   &  \locVold \\
\end{tabular}
%---------------------------------
\end{center}
\vminfive
\mycaption[Integration (LOC)]{tab-loc}{Integration (LOC)}{The table
quantifies our efforts in integrating \sck.
%
Riak's integration
is smaller due to the conciseness of Erlang.
%
Voldemort's integration does not include PIL for the old
bug we reproduced
}
\vminfive
\end{table}




We integrate our \sck methodology to three popular distributed key-value
stores: Cassandra \cite{Lakshman+09-Cassandra}, Riak \cite{RiakWeb}, and
Voldemort \cite{VoldemortWeb}.
%
The implementation of \sck involves two parts:
%
{\bf (a)} changes to make the target system ``\sck-able'' on one machine and
%
{\bf (b)} the \ts{sck} tool itself (specifically: \scass, \svold, and
\sriak).\footnote{As an analogy, part (b) is similar to specific \ts{fsck}s
  (\eg, \ts{fsck.ext3}, \ts{fsck.xfs}) while part (a) is
  similar to how file system code is modified to make \ts{fsck}
  fast \cite{Henson+06-Chunkfs, Ma+13-Ffsck}.}
% LOC
Part (a) involves
the integration of SPC, MFR, and GEDA, as well as PIL interpositioning
points.  Part (b) is the code for pre-memoization, time profiling, and
other \ts{sck} setups.
%
Table \ref{tab-loc} quantifies the integration efforts.  
%Overall, our modifications add 2-4\% of code that provide a new powerful
%functionality.
%

\vni {\bf Generality:} We show the generality of \sck with two major efforts.
First, we picked three systems and scale-checked various control-path
protocols within them, for a total of \numProt\ protocols:
%
\numProtCass\ Cassandra (bootstrap, scale-out, decommission),
%
\numProtRiak\ Riak (bootstrap+rebalance), and 
%
\numProtVold\ Voldemort (rebalancing) protocols.
%
A protocol can be built on top of other protocols (\eg, bootstrap on
gossip and failure detection protocols).


Second, we migrated \sck to a total of \numVers\ old and new releases:
%
\numVersCass\ Cassandra (v0.8.9, v1.1.10, v1.2.0, v1.2.9, v2.2.5),
%
\numVersRiak\ Riak (v0.14.2, v2.1.3), and
%
\numVersVold\ Voldemort (v0.90.1, v1.10.21).
%
This effort is also important to show how \sck can find old and new
bugs. 



\vni {\bf Simplicity:}
%
Table \ref{tab-loc} shows \sck\ requires thousands of LOC, which we
believe is a justified cost for supporting scale-checkability.  We want to
emphasize that this is a {\em one-time} cost; subsequent migrations are
fast.  Our first complete integration to Cassandra (v1.2.9) took almost a
year; we needed to learn from scratch about Cassandra and its scalability
bugs and also design \sck.  However, after \sck techniques were solid,
migration to other versions (Cassandra-v0.8.9, v1.1.10, v2.2.5) only took
{\em 1~week} each.  Next, our first integration to Riak (v0.14.2) only
took {\em 4~weeks} (although Riak was completely new to us).  A subsequent
migration (Riak-v2.1.3) only took {\em 4~days}.  The efforts for
Voldemort is also similar.
%
Overall, we expect \sck integration can be done more seamlessly with
today's DevOps practice \cite{Limoncelli+11-Devops}, where developers are
testers and testers are developers.







\if 0
All \sck techniques are straightforward to integrate.  Below are more
specific integration items we did not discuss before.
\hsg{list of all problems, existing systems are not designed
with scale checkability in mind.}
- Global lock
- Small size thread pool
- Bounded queue
\fi

\if 0
\vni {\bf Pre-Memoization:} During the pre-memoization step (with Naive
Packing), failure-detection or all timeout-related logic has to be
converted to use logical time.  This is important to remove false
positives; for example, a node might not hear any new gossip of another
node after 10 seconds, but this is due to CPU contention (thread queueing
delay) in Naive Packing.  This is a similar concern in DieCast \cite{x},
hence the time-dilation support in their VMM.  In our case, we manually
modify the application ... \hsg{how??}
\fi









\if 0
{\bf Limitations:}
\hsg{define the target protocols. the target protocol, 
the target problem, to a methodology. many P2P systems have this problem.
it cannot debug bugs that are not caused by cascading CPU problems.
There are many scalability bugs.  Just like Exalt cannot be used for our
bugs, ours cannot be used for their bugs.  At the end Sck methodology
will grow more.  So far we are limited to one machine.
The limit is the core of the one machine. To scale up we can have
more cores (co-processors with large DRAM). 
%
Also why note 10 machines? why one machine?
SCk can be extended to 10 machines, but then we need to implement
batching, because in reality each node only receives XX messages per second,
but with 50 nodes, it receives 50x messages per second.
ONe must look into this effect.
} 
\fi


