
% ------------------------
\subsubsection{Local-Message Independence (LMI)}
\label{sam-lmi}

% LMI basic
We define {\em local messages} as messages that are concurrently in
flight to a given node (\ie, intra-node messages).  As shown in
Figure~\ref{fig-pol}a, a black-box DPOR treats the message processing
semantic inside the node as a black box, and thus must declare the
incoming messages as dependent, leading to 4! permutation
of \ma\mb\mc\md.  On the other hand, with white-box knowledge,
local-message independence (LMI) can define {\em independency
relationship among local messages}.  For example, illustratively in
Figure~\ref{fig-pol}b, given the node's local state (\ls) and the
processing semantic (embedded in the \ts{if} statement), LMI is able
to define that \ma\ and \mb\ are dependent, \mc\ and \md\ are
dependent, but the two groups are independent, which then leads to
only 4 re-orderings.  This reduction illustration is similar to the
one in Section~\ref{mot-state}, but this time LMI enables DPOR
application on local messages.


% LMI question
LMI can be easily added to a dmck.  A dmck server typically has a
complete view of the local states (\sec\ref{mot-bgterms}).  What is
needed is the {\em message processing semantic}: how will a node (\nn)
process an incoming message (\mm) given the node's current local state
(\ls)?  The answer lies in these four simple {\em message processing
patterns} (discard, increment, constant, and modify):


{\small
\begin{alltt}
     \underline{Discard:}           \underline{Increment:}  
     if (pd(m,ls))      if (pi(m,ls))
      (noop);             ls++;       

     \underline{Constant:}          \underline{Modify:}  
     if (pc(m,ls))      if (pm(m,ls))
       ls = Const;        ls = modify(m,ls);
\end{alltt}
}

% LMI pattern .....
In practice, \ls\ and \mm\ contain many fields.  For simplicity, we
treat them as integers.  The functions with prefix \pp\ are boolean
read-only functions (predicates) that compare an incoming message
(\mm) with respect to the local state (\ls); for example, \pd\ can
return true if \ts{m<s}.  The first pattern is a {\em discard} pattern
where the message is simply discarded if \pd\ is true.  This pattern
is prevalent in distributed systems with votes/versions; old
votes/versions tend to be discarded (\eg, our example
in \sec\ref{sam-ex}).  The {\em increment} pattern performs an
increment-by-one update if \pi\ is true, which is also quite common in
many protocols (\eg, counting commit acknowledgements).  The {\em
constant} pattern changes the local state to a constant whenever \pc\
is true.  Finally, the {\em modify} pattern changes the local state
whenever \pm\ is true.


% LMI policy
Based on these patterns, we can  apply LMI in the following
ways.
%
(1) \mx\ is independent of \my\ if \pd\ is true on any of \mx\ and \my.
That is, if \mx\ (or \my) will be discarded, 
then it does not need to be re-ordered 
with other messages.
%
(2) \mx\ is independent of \my\ if \pi\ (or \pc) is true on both \mx\ and \my.
That is, the re-orderings do not matter because the local state is
monotonically increasing by one (or changed to the same constant).
%
(3) \mx\ and \my\ are dependent if \pm\ is true on \mx\ and 
\pd\ is not true on \my.  
That is, since both messages modify the local state in unique ways, then the
re-orderings can be ``interesting'' and hence should be exercised.
%
All these rules are continuously evaluated before every event is
enabled.  If multiple cases are true, dependency has higher precedence
than independency.







% LMI, application 
Overall, LMI allows dmck to smartly skip redundant re-orderings by
leveraging simple patterns.  The job of the tester is to find the
message processing patterns from a target protocol and write {\em
protocol-specific rules} (\ie, filling in the content of the four LMI
predicate functions (\pd, \pi, \pc, and \pm) specific to the target
protocol).  As an example, for our simple leader election protocol
(\sec\ref{sam-ex}), \pd\ can be as simple as:
return \ts{m.vote} \ts{<}
\ts{ls.myVote}.


