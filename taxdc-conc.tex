
% -----------------------------------------------
\section{Conclusion}

In this chapter, we have shown our in-depth study result on DC bugs and
introduce \taxdc, the largest and most comprehensive taxonomy of DC bugs. 
\taxdc\ categorizes DC bugs in three major aspects including triggering, error
and failure, and fix. This chapter show some complexity of DC bugs such as:

\begin{itemize}
% surface
\item We see a lack of effective testing, verification, and analysis tools to
detect DC bugs during development process.

% protocols
\item Many DC bugs hide in complex concurrent executions of {\em multiple}
protocols and not only user-facing foreground protocols, but also background and
operational.

% faults
\item Majority of DC bugs surface in the presence of hardware faults such as
machine crashes (and reboots), network delay and partition (timeouts), and disk
errors.  

% silent
\item Half of DC bugs lead to silent failures and hence are hard to debug in
production and reproduce offline.

\end{itemize}

% ---------------------------------------------

However, our results also bring fresh and positive insights:

\begin{itemize}
% trigger-shan
\item From the triggering patterns, we see opportunities to build DC bug
detection to focus on timing-specification inference and violation detection.

% err-shan
\item Half of DC bugs lead to {\em explicit} local or global errors which allow
inferring timing specifications based on local correctness specifications, in
the form of error checking already provided by developers.

%fix-shan
\item Most DC bugs are fixed through a small set of strategies and some are
simple which implies research opportunities for automated in-production fixing
for DC bugs.

% highlights
%\item Many other observations are made that enable us to analyze the gap between
%state-of-the-art tools and real-world DC bugs as well as between research in LC
%and DC bugs.

\end{itemize}

We believe that our observations in this chapter will be a foundation to help us
advance the state of the art to combat DC bugs.
