



\section{Errors and Failures}
\label{sec-err}

\subsection{Error Symptoms}
\label{err-err}

% goal
From the triggering conditions, we then scrutinize the {\em first}
error that happens immediately after.  First errors are the pivotal
point that bridges the triggering and error-propagation process.
Identifying first errors help failure diagnosis get closer to
disclosing bug triggering and root causes and help bug detection get
closer to accurately predict failures.




% our category
We categorize first errors into {\em local} errors and {\em global}
errors, based on whether they can be observed from the triggering node
\nt\ alone.  Here, \nt\ is the node where triggering ends.  It is the
receiver node of untimely messages (\eg, node C in Figure \ref{pat}a)
or the node with untimely fault (\eg, node A in Figure \ref{pat}i).
For each error, we also check whether it is an {\it explicit} or {\em
  silent} error.  
Table \ref{tab:error} and Figure \ref{bars}l
(\BERR) show the per-system and overall breakdowns
respectively. 
Some MapReduce bugs caused multiple concurrent first errors of different types.

\if 0
\jirafootnote{err-err}{tab:error}{
\spa \ca{5925}, \hb{6375}, \mr{5198};  % local memory exception
\spb \hb{4540}, \mr{4607}, \zk{1046};           % local state machine/semantic
\spc \hb{5606}, \mr{4252}, \zk{1144};           % local hangs 
\spd \ca{2590}, \hb{6227}, \mr{4099}, \zk{1382};    % local silent (other than hang)
\spe \ca{1011}, \hb{4015}, \mr{2953};  % global wrong
\spf \ca{5393}, \hb{6060}, \mr{4819};  % global miss
\spg \ca{6023}, \hb{6070}, \zk{962}; % global silent
}
\fi


% local
First, DC bugs can manifest into both local explicit errors and local
silent errors. The former includes
%
{\em memory exceptions} such as null-pointer exceptions
(\pctErrLocMem\ in Table \ref{tab:error}) and {\em semantic errors} 
such as wrong state-machine transition exceptions thrown by 
the software (\pctErrLocSem).
%
Local silent errors include
%
{\em hangs}, such as forever waiting for certain states to change
or certain messages to arrive which are typically observed implicitly
by users (\pctErrLocHang),
%
and {\em local silent} state corruption, such as half-cleaned
temporary files (\pctErrLocSil).




\begin{table}[t]
\small
\centering
\begin{tabular}{lcccccccc}
\toprule
   & \multicolumn{4}{c}{Local Errors} && \multicolumn{3}{c}{Global Errors}\\
\cmidrule{2-5}
\cmidrule{7-9}
   & Mem & Sem & Hang & Sil &    & Wrong & Miss & Sil \\
%   & \spa & \spb & \spc & \spd &    & \spe & \spf & \spg \\
\midrule
CA &  2  & 0 & 0 & 4 & & 3 & 3 & 7  \\
HB &  1  & 2 & 1 & 2 & & 15 & 3 & 6  \\
MR &  2  & 13 & 7 & 4 & & 14 & 4 & 0  \\
ZK &  0  & 6 & 2 & 5 & & 1 & 0 & 5  \\
\midrule
All &  5  & 21 & 10 & 15 & & 33 & 10 & 18  \\
\bottomrule
\end{tabular}
\mycaption{tab:error}{First error symptoms of DC bugs (\sec{\ref{err-err}})}
{Some bugs cause multiple concurrent first errors.}
\end{table}


\if 0
The
abbreviations are the following.
Mem: memory error, Sem: semantic,
Sil: silent,  Miss: missing message, and Wrong: wrong message.
\fi


% global
When local error is non-obvious in \nt, we analyze if the error is
observable in other nodes communicating with \nt.  Many
DC bugs manifest into explicit global errors through {\em wrong
messages} (\pctErrGlobWrong\ in Table \ref{tab:error}). Specifically, 
the communicating
node receives an incorrect message from \nt, and throws an exception
during the message handling.  However, a few DC bugs still lead to
silent global errors. These include {\em missing messages}, where
\nt\ never sends a reply that the communicating node is waiting for
in the absence of timeout (\pctErrGlobMiss), and {\em global
  silent} state corruption such as replica inconsistencies
between \nt\ and the other nodes (\pctErrGlobSil).


\finding{\textit{Local} and \textit{global} first errors are about
equally common; \pctErrLoc\ vs. \pctErrGlob\ in Figure \ref{bars}m
(\BLG).  About half of the DC bugs generate \textit{explicit} first
errors (\pctErrExp), including local exceptions and global wrong
messages, and the remaining DC bugs lead to \textit{silent} errors
(\pctErrImp), as shown in Figure \ref{bars}n (\BES). 
%
Some of them immediately lead to hangs in the triggering node $N_T$
(\pctErrLocHang) or a node communicating with $N_T$
(\pctErrGlobMiss).  }








% =========================================
\subsection{Failure Symptoms}
\label{err-fail}






Figure \ref{bars}o (\BFAIL) shows that errors from DC bugs will
eventually lead to a wide range of fatal failures including
%
node downtimes (\pctFailNode),
data loss/corruption/inconsistencies (\pctFailData),
operation failures (\pctFailOp),
and performance degradation (\pctFailPer).
%
A node downtime happens when the node either crashes or
hangs (\ie, it may still be heartbeating).
It happens to both master/leader nodes and worker/follower nodes
in our study.
%Developers often have a hard
%time in debugging fail-silent behaviors and prefer fail-stop
%behaviors \cite{mr-3634}.
%
Data-related failures and performance problems are an artifact of
incorrect state logic induced from DC bugs.  For example, in HBase,
concurrent region updates and log splittings can cause data loss.
In Cassandra, some dead nodes are incorrectly listed as alive causing
unnecessary data movement that degrades performance.
% 
Node downtimes and data-related failures could also cause some operations
to fail.  To avoid double counting,
we consider a bug as causing operation failures only when it does not cause
node downtimes or data-related failures.

\if 0
\jirafootnotable{err-fail}{
\spa \hb{6153}, \mr{5198}, \zk{1144}; % node down
\spb \ca{2083}, \hb{6227}, \mr{4819}, \zk{1154}; % data loss / stale / corrupt
\spc \ca{5631}, \hb{5780}, \mr{4833}, \zk{1291}; % opfail
\spd \ca{3626}, \hb{5606}, \zk{975}; % perf degrade
}
\fi






