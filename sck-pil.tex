

\subsection{Processing Illusion (PIL)}
\label{sc-pil}

While I/O and memory bottlenecks are often blamed for many scalability problems
\cite{Ousterhout+15-MakingSense,Konstantin+10-HDFSScalability, Wang+14-Exalt},
scalability bugs also could be caused by cascading impacts of CPU-intensive
processing as the example of Cassandra bugs.
%
To emulate CPU-intensive processing, we introduce {\em processing illusion}
(PIL), an approach that {\em replaces an actual processing with \sleep}.  For
example, in the sample Cassandra bug, we can replace the expensive ring-table
update with \ts{sleep(t)} where \ts{t} is an accurate timing of how long the
update takes.


The intuition behind PIL is similar to the intuition behind other
emulation techniques.
%
For example, Exalt provides an illusion of storage space; their insight
was ``how data is processed is  not affected by the content of the data
being written, but only by its size'' \cite{Wang+14-Exalt}.
%
PIL provides an illusion of compute processing; our insight is that
{\em ``the
  key to computation is not the intermediate results, but rather the
  execution time and eventual output''.}
% {\em ``how
%   compute matters is not because of the intermediate computation, but 
% rather by its execution time and output.''}
%

PIL might sound outrageous in the first place, but it is feasible only if 
the following concerns are addressed:
%
% replace compute with sleep
%
\sec\ref{sc-pil-1}: How can a function be safely replaced
with \sleep \textit{without} changing the whole processing semantic?
And how to find specific functions that should be replaced with \sleep?
%
\sec\ref{sc-pil-3}: How can we produce the output if the actual compute is
skipped?
%
\sec\ref{sc-pil-4}: How can we predict the actual compute time (\ts{t})
accurately?

\subsubsection{PIL-Safe Functions}
\label{sc-pil-1}

Our first challenge is to ensure that functions (or code blocks) can be safely
replaced with \sleep, but still retain the cluster-wide behavior and unearth the
bug symptoms.  We name such functions as {\em PIL-safe functions}.
%
We identify two main characteristics of sleep-safe functions:
% by memoization
\begin{enumerate}
\item Memoizable output: a PIL-safe function must have a memoizable
(deterministic) output based on the input of the function.
% ...
\item Non-pertinent I/Os: if a function performs disk/network I/Os that are not
pertinent to the correctness of the corresponding protocol, the function is
PIL-safe.  For example, in the sample Cassandra bugs, there is a ring-table
checkpoint (not shown) needed for fault tolerance but is irrelevant (never read)
during bootstrapping.
%
\end{enumerate}
%We extend \sfind to \sfindp, which includes a static analysis that finds code
%blocks in scale-dependent loops that can be safely PIL-ed.
Depending on the modularity of the target system, manually finding such target
functions can be challenging, primarily because scale-dependent nested loops can
span across multiple functions. Right now, we rely on developers to identify
such that functions, and we leave automatic PIL-taking function extracting for
future work (Chapter \ref{chp-con}).

% -----------------------------------
\subsubsection{Pre-Memoization with Order Determinism}
\label{sc-pil-3}

As sleep-safe functions no longer perform the actual computation, the next
question to address is: how do we manufacture the output?  We find there
are sleep-safe functions with non-pertinent outputs
(\sec\ref{sc-pil-1}). For these functions, only time profiling is needed
(\sec\ref{sc-pil-4}) but not output recording.  However, there are also
sleep-safe functions with non-pertinent intermediate data but with outputs
that are needed.
%
For this latter case, we need to manufacture the outputs such that the
global behavior is not altered (\eg, cluster bootstrapping or rebalancing
should terminate successfully).
%
Our solution is {\em pre-memoization}: given a sleep-safe
function, we identify all the possible inputs, and for every input, run
the function and pre-memoize the output.  
% This is done prior to \sck; 
When \sck runs, it will use the pre-memoized outputs.


Unfortunately, pre-memoization in the context of large-scale,
decentralized, non-deterministic distributed systems requires an {\em
  ``infinite''} time and storage space.  The issue is that the state of
each node (the input) depends on the {\em order} in which messages arrive
(which can be random).
%
As an example, let's consider Riak's bootstrap+rebalance protocol where
eventually all nodes own a similar number of partitions.  
% The decentralized algorithm is quite complex \cite{algorithmOnline}, but
% put simply, each node
A node initially has an unbalanced partition table, receives another
partition table from a peer node, then inputs it to a rebalance function,
and finally sends the output to a {\em random} node via gossiping.  {\em
  Every} node repeats the same process until the cluster is balanced.
%
In a Riak cluster
with $N$$=$256 and 
$P$\footnote{$P$: the number of key-partitions per node;  
A key-partition is typically a random integer 
within 2$^{64}$ keyrange.}$=$64, there are in total 2489 rebalance iterations
with a set of specific inputs in {\em one} run.  
{\em Another} run of the protocol will
result in a {\em different} set of inputs due to gossip randomness.
%, due to the gossip randomness.
Our calculation shows that there are 
$(N^{NP})^2$ 
possible inputs.
% ; with 550-KB
% partition table, this pre-memoization requires \xxx Exabytes of space.




To address this problem, we pre-memoize with {\em order determinism}.
Thus, repeated runs of the same workload in \sck mode will use the {\em
  same} global message ordering (akin to deterministic record and replay
\cite{Geels+07-Friday}).
% Gautam+09-ODR,   Geels+06-Liblog, Guo+08-R2, Soyeon+09-PRES
%
For example, across different runs, a Riak node now receives gossips from
the same sequence of nodes.
%
With order determinism, pre-memoization and \sck work as follow: 
%
% {\bf (1)} We first run the whole cluster on a real deployment and interpose
% sleep-safe functions.
\begin{enumerate}
\item We run all nodes on one machine
without PIL (more details in \sec\ref{sc-summ}) and interpose
sleep-safe functions.
%
\item When sleep-safe functions are executed, we record the inputs and
corresponding outputs to a {\em memoization database} (SSD-backed files).
%
\item During this pre-memoization phase, 
we {\em record message non-determinism} (\eg,
gossip send-receive pairs and their orderings).
%
\item After pre-memoization completes, we can 
repeatedly run \sck wherein order
determinism is enforced (\eg, no randomness), sleep-safe functions
replaced with PIL, and their outputs retrieved from the memoization
database.
%
%Note that steps 1-3 are the only steps that require real deployment.  
\end{enumerate}
We omit some details above but will summarize the steps again later along
with other features (\sec\ref{sc-summ}).


%
With order determinism, the memoization database is kept small as we only
record the possible inputs within {\em one} deterministic order.  In the
256-node Riak's case above, the database only needs to store around 2500
input-output pairs (the number of rebalance iterations observed) in 1.3 GB
of memoized data (and 5.3 GB for the 512-node setup).
% remove if no space
We also note that while the idea of deterministic systems has been made
popular recently,
% \cite{Aviram+10-Determinator, Bergan+10-dOS}, 
the concept of deterministic distributed systems is still not practical
due to the excessive runtime overhead (\eg, 10x slower
\cite{Hunt+13-DDOS}).  However, in the context of offline methodology such
as \sck, order determinism can be exploited in a fitting manner.




% -----------------------------------
\subsubsection{Time Profiling}
\label{sc-pil-4}

As sleep-safe functions are replaced with \ts{sleep(t)}, we need to
accurately predict the actual compute time (\ts{t}).  There are two
different approaches we take, depending on the target protocols.
%
\begin{enumerate}
\item The first approach is to profile compute time in situ with
the pre-memoization phase ({\em in-situ time profiling}).  
That is, for each input observed during
pre-memoization, we also record how long the processing takes.
%
\item Another approach is to profile compute time with an {\em offline
time  profiling}, which is feasible for functions with non-pertinent outputs
(\sec\ref{sc-pil-1}), which do not need pre-memoized outputs.  
\end{enumerate}

With offline time profiling, we simply profile the expensive function
exclusively by itself with the possible input space.
If faster profiling is needed, we can sample the input space.
For example, in the sample Cassandra bug,
the expensive function depends on a 2-dimensional input (\#commit states
and current ring table size), each ranges from 1 to $N$.  
With $N^2$ profiles, the profiling time can
take more than 
one day without sampling when $N$ is large (\eg, 512 nodes).  When \sck
runs, given an input, we normalize \ts{t} based on the sampled profile.

% ......
We now address two further questions.  First, is time profiling necessary?
In other words, is static prediction sufficient (\eg, extrapolation based
on a \ts{for}-loop timing)?  In our case, static prediction is hard
to achieve; nested loops can span across multiple functions with many
\ts{if-else} conditions.  For example, state-update processing in
the sample Cassandra bug can range from 0.001 to 4 seconds depending on the
multi-dimensional input (\sec\ref{mot-observe}).  
%
% \hsg{why don't we prove this by saying, we did this with static prediction,
% but we got inaccurate results. TODO??}

% Hence, time profiling is needed for accuracy.

% remove if no confirmation from Cesar \hsg{Cesar??} 

% Furthermore, processing time depends on CPU speed and storage latency.
% We observed a case of a customer deploying Riak in ``weak'' machines,
% which surfaces a scalability bug that the developers did not expect
% \cite{riak?}.  Thus, dynamic profiling time must be based on a similar
% machine deployed by the customer who reported the bugs.



% predict scalability bugs from profiling
Second, is it obvious from the profiled time that a scalability bug will
appear, hence obviating the need for \sck?  
%
As suggested earlier, every implementation is unique
(\sec\ref{mot-observe}).  In the sample Cassandra bug for example, {\em if} Cassandra
processes gossips in a multi-threaded manner, long processing time would
not lead to the scalability bug.
%
In fact, patches for scalability bugs do not always remove the expensive
computation.
%
Scalability bugs are not merely about the expensive functions, but rather
their potential cascading impacts, hence it is essential to run \sck 
in addition to time profiling.
% not extrapolation
We want to emphasize that our profiling approach is not the same as
extrapolation, which tends to stop profiling at a certain 
scale (\eg, 100 nodes)
and extrapolates the behavior for larger scales.  Our
profiling and \sck phases run at real scales.
% the same number of nodes.


\vfive Overall, PIL significantly removes processing contention and
reduces CPU utilization.  Interestingly, as we colocate more nodes, before
we hit 100\% CPU utilization, we hit other major colocation bottlenecks
such as memory exhaustion and process/thread context-switching delays.
%
For this reason, we re-architect our target systems to make them
scale-checkable with the next three optimization techniques
(\sec\ref{sc-spc}-\sec\ref{sc-mem}).



\if 0
\hsg{new:}
Exalt \cite{exalt} provides a compression technique that is powerful for
cases where storage capacity is the colocation bottleneck.
The compression technique however increases CPU utilization
and thus does not solve CPU-bottleneck colocation problem.
For example, it admits that the CPU-intensive HBase region servers
cannot benefit from Exalt colocation. 
\fi
