
\section{TaxDC}
\label{sec-taxdc}

In this formal study on DC bugs, we do in-depth analysis of \numDcBugs\ DC
bugs.  The bugs came from four popular cloud distributed systems: Cassandra
\cite{CassandraWeb}, HBase \cite{HBaseWeb}, Hadoop MapReduce \cite{HadoopWeb},
and ZooKeeper \cite{ZooKeeperWeb}.
%
We introduce \taxdc, a comprehensive taxonomy of real-world DC bugs across
several axes of analysis such as the triggering timing condition and input
preconditions, error and failure symptoms, and fix strategies, as shown in
detail in Table \ref{tab:tax}.

As the main contribution, \tdc\ will be the first large-scale DC-bug benchmark.
In the last six years, bug benchmarks for LC bugs have been released
\cite{Jalbert11-RADBench, jieyu}, but no large-scale benchmarks exist for DC
bugs.  Researchers who want to evaluate the effectiveness of existing or new
tools in combating DC bugs do not have a benchmark reference.  \tdc\ provides
researchers with more than 100 thoroughly taxonomized DC bugs to choose from.
Practitioners can also use \tdc\ to check whether their systems have similar
bugs.  The DC bugs we studied are considerably general, representing bugs in
popular types of distributed systems.

As a side contribution, \tdc\ can help open up new research directions. In the
past, the lack of understanding of real-world DC bugs has hindered researchers
to innovate new ways to combat DC bugs.  The state of the art focuses on three
lines of research: monitoring and postmortem debugging \cite{Geels+07-Friday,
Liu+08-D3S, Liu+07-WiDS, Reynolds+06-Pip}, testing and model checking
\cite{Guo+11-Demeter, Killian+07-LifeDeathMaceMC,
Simsa+10-Dbug, Yang+09-Modist}, and verifiable language frameworks
\cite{Desai+13-PLang, Wilcox+15-Verdi}.  We hope our study will not only improve
these lines of research, but also inspire new research in bug detection tool
design, runtime prevention, and bug fixing, as elaborated more in Section
\ref{sec-less}.

